\documentclass{article}

\usepackage{tabularx}
\usepackage{booktabs}

\title{Problem Statement and Goals\\Pole Balancing}

\author{Mina Mahdipour}

\date{January 2023}

\input{}

\begin{document}

\maketitle

\begin{table}[hp]
\caption{Revision History} \label{TblRevisionHistory}
\begin{tabularx}{\textwidth}{llX}
\toprule
\textbf{Date} & \textbf{Developer(s)} & \textbf{Change}\\
\midrule
20 Jan. 2023 & Mina Mahdipour & First version of document\\
25 Jan. 2023 & Mina Mahdipour & Changes in problem and input-output sections\\
\bottomrule
\end{tabularx}
\end{table}

\section{Problem Statement}

An inverted pendulum is a pendulum upside down with it pivot point under its center of mass. it is unstable and without additional help and by any small disturbances will fall over. To stay upright, it needs a controller to keep its center of mass above its pivot point even when disturbances occur. The objective of the control system is to balance the inverted pendulum by applying a force to the place that the pendulum is attached to.

Inverted Pendulums are Everywhere, from  the human posture systems, Segway, to the launching of a rocket.

\subsection{Problem}

Stabilizing One unstable pole mounted on a cart that can move horizontally is the main problem. There are two degrees of freedom. The horizontally movement of cart and the pendulum angle with vertical axis. Hence there are two equations of motion, one for each. Solving these equations describe the motion needed on the cart to keep the whole system at the equilibrium point.

\subsection{Inputs and Outputs}

The inputs of the control system will be the external force to the cart which moves it, and the outputs are the angular position of the pendulum, and the horizontal position of the cart. Although there are lots of parameters such as mass of the card, mass of the pendulum, pendulum length and so on which we consider as constant values.

\subsection{Stakeholders}

As the inverted pendulum is widely used as a benchmark for testing control strategies and algorithms, and it can be used in different applications, so the potential stakeholders could be all the related industries. On the other hand, it is a classic problem in dynamics and control theory, so the final system can be used in high school as an educational material.

\subsection{Environment}

There is no limitation and the final software will be compatible with Windows, MacOS and Linux operating systems.

\section{Goals}

\begin{enumerate}
  \item	Designing a software system that can control and stabilize the pendulum.
  \item Verifying that the controller can handle disturbances.
  \item	Measuring the effect of the different parameters.
\end{enumerate}
\section{Stretch Goals}
\begin{enumerate}
  \item Visualizing the problem with an interactive graphic design.
  \item Making a real model includes a rod with a weight on the bottom and the controller.
  \item Simulating double inverted pendulum problem.
\end{enumerate}

\end{document}
