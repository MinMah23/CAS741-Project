% THIS DOCUMENT IS TAILORED TO REQUIREMENTS FOR SCIENTIFIC COMPUTING.  IT SHOULDN'T
% BE USED FOR NON-SCIENTIFIC COMPUTING PROJECTS
\documentclass[12pt]{article}

\usepackage[square,numbers]{natbib}
\usepackage{amsmath, mathtools}
\usepackage{amsfonts}
\usepackage{amssymb}
\usepackage{graphicx}
\usepackage{colortbl}
\usepackage{xr}
\usepackage{hyperref}
\usepackage{longtable}
\usepackage{xfrac}
\usepackage{tabularx}
\usepackage{float}
\usepackage{siunitx}
\usepackage{booktabs}
\usepackage{caption}
\usepackage{pdflscape}
\usepackage{afterpage}

\usepackage[round]{natbib}

%\usepackage{refcheck}



\input{}

% For easy change of table widths
\newcommand{\colZwidth}{1.0\textwidth}
\newcommand{\colAwidth}{0.13\textwidth}
\newcommand{\colBwidth}{0.82\textwidth}
\newcommand{\colCwidth}{0.1\textwidth}
\newcommand{\colDwidth}{0.05\textwidth}
\newcommand{\colEwidth}{0.8\textwidth}
\newcommand{\colFwidth}{0.17\textwidth}
\newcommand{\colGwidth}{0.5\textwidth}
\newcommand{\colHwidth}{0.28\textwidth}

% Used so that cross-references have a meaningful prefix
\newcounter{defnum} %Definition Number
\newcommand{\dthedefnum}{GD\thedefnum}
\newcommand{\dref}[1]{GD\ref{#1}}
\newcounter{datadefnum} %Datadefinition Number
\newcommand{\ddthedatadefnum}{DD\thedatadefnum}
\newcommand{\ddref}[1]{DD\ref{#1}}
\newcounter{theorynum} %Theory Number
\newcommand{\tthetheorynum}{T\thetheorynum}
\newcommand{\tref}[1]{T\ref{#1}}
\newcounter{tablenum} %Table Number
\newcommand{\tbthetablenum}{T\thetablenum}
\newcommand{\tbref}[1]{TB\ref{#1}}
\newcounter{assumpnum} %Assumption Number
\newcommand{\atheassumpnum}{P\theassumpnum}
\newcommand{\aref}[1]{A\ref{#1}}
\newcounter{goalnum} %Goal Number
\newcommand{\gthegoalnum}{P\thegoalnum}
\newcommand{\gsref}[1]{GS\ref{#1}}
\newcounter{instnum} %Instance Number
\newcommand{\itheinstnum}{IM\theinstnum}
\newcommand{\iref}[1]{IM\ref{#1}}
\newcounter{reqnum} %Requirement Number
\newcommand{\rthereqnum}{P\thereqnum}
\newcommand{\rref}[1]{R\ref{#1}}
\newcounter{nfrnum} %NFR Number
\newcommand{\rthenfrnum}{NFR\thenfrnum}
\newcommand{\nfrref}[1]{NFR\ref{#1}}
\newcounter{lcnum} %Likely change number
\newcommand{\lthelcnum}{LC\thelcnum}
\newcommand{\lcref}[1]{LC\ref{#1}}

\usepackage{fullpage}

\newcommand{\deftheory}[9][Not Applicable]
{
\newpage
\noindent \rule{\textwidth}{0.5mm}

\paragraph{RefName: } \textbf{#2} \phantomsection 
\label{#2}

\paragraph{Label:} #3

\noindent \rule{\textwidth}{0.5mm}

\paragraph{Equation:}

#4

\paragraph{Description:}

#5

\paragraph{Notes:}

#6

\paragraph{Source:}

#7

\paragraph{Ref.\ By:}

#8

\paragraph{Preconditions for \hyperref[#2]{#2}:}
\label{#2_precond}

#9

\paragraph{Derivation for \hyperref[#2]{#2}:}
\label{#2_deriv}

#1

\noindent \rule{\textwidth}{0.5mm}

}

\begin{document}

\title{Software Requirements Specification for: Simulating the Inverted Pendulum system} 
\author{Mina Mahdipour}
\date{\today}
\maketitle

~\newpage

\pagenumbering{roman}

\tableofcontents

~\newpage

\section*{Revision History}

\begin{tabularx}{\textwidth}{p{3cm}p{2cm}X}
\toprule {\bf Date} & {\bf Version} & {\bf Notes}\\
\midrule
February 1, 2023  & 1.0 & First version of document\\
February 3, 2023  & 1.1 & Complete the Description part\\
February 4, 2023  & 1.2 & Update Requirements part\\
February 6, 2023  & 1.2 & Update Specific System Description part\\

\bottomrule
\end{tabularx}

~\newpage

\section{Reference Material}

This section records information for easy reference.

\subsection{Table of Units}

Throughout this document SI (Syst\`{e}me International d'Unit\'{e}s) is employed as the unit system.  In addition to the basic units, several derived units are used as described below.  For each unit, the symbol is given followed by a description of the unit and the SI name.
~\newline

\renewcommand{\arraystretch}{1.2}
%\begin{table}[ht]
  \noindent \begin{tabular}{l l l} 
    \toprule		
    \textbf{symbol} & \textbf{unit} & \textbf{SI}\\
    \midrule 
    \si{\kilogram} & mass & kilogram\\
    \si{\newton} & force & newton\\
    \si{\deg} & angle & degree\\
    \si{\metre} & distance & metre\\
    \si{\tau} & torque & newton.metre\\
    \bottomrule
  \end{tabular}
  %	\caption{Provide a caption}
%\end{table}

\subsection{Table of Symbols}

The table that follows summarizes the symbols used in this document along with
their units. The symbols are listed in alphabetical order.

\renewcommand{\arraystretch}{1.2}
%\noindent \begin{tabularx}{1.0\textwidth}{l l X}
\noindent \begin{longtable*}{l l p{12cm}} \toprule
\textbf{symbol} & \textbf{unit} & \textbf{description}\\
\midrule
$b$ & dimensionless & coefficient of friction for the cart\\ 
$d_py$ & \si{metre}& distance traveled by the center of mass of pendulum in {y} direction\\
$d_px$ & \si{metre}& distance traveled by the center of mass of pendulum in {x} direction\\

$g$ & \si {\metre\per\square\second} & Acceleration due to gravity\\
$F(t)$ & \si {\newton} & Horizontal force exerted on the cart \\
$I$ & \si {\kilogram\square\metre} & moment of inertia of the pendulum\\

$\ell$ & \si{\metre} & Length to pendulum center of mass\\ 
$M_c$ & \si{kilogram} & Mass of cart\\
$m_p$ & \si{kilogram} & Mass of pole\\
$H$ & \si {\newton} & Horizontal reaction force of the cart or the pole\\
$P$ & \si {\newton} & Vertical reaction force component of the cart or the pole\\
$x(t)$ & \si{metre} & The distance which cart transfer during time (t) \\
$\dot{x}$ & \si {\metre\per\second} & Velocity of cart\\
$\ddot{x}$ & \si {\metre\per\square\second} & Acceleration of cart\\  
$\theta (t)$ & \si{\deg} & Pendulum angle from vertical (down)  \\
$\dot{\theta}$ & \si {\deg\per\second} & Angular velocity of the pendulum\\
$\ddot{\theta}$ & \si {\deg\per\square\second} & Angular acceleration of the pendulum\\
 
\bottomrule
\end{longtable*}

\subsection{Abbreviations and Acronyms}

\renewcommand{\arraystretch}{1.2}
\begin{tabular}{l l} 
  \toprule		
  \textbf{symbol} & \textbf{description}\\
  \midrule 
  A & Assumption\\
  DD & Data Definition\\
  GD & General Definition\\
  GS & Goal Statement\\
  IM & Instance Model\\
  LC & Likely Change\\
  PS & Physical System Description\\
  R & Requirement\\
  SRS & Software Requirements Specification\\
  IP & Simulating the Inverted Pendulum\\
  T & Theoretical Model\\
  \bottomrule
\end{tabular}\\

\subsection{Mathematical Notation}

{In this document, we do not use any specific mathematical notation.}

\newpage

\pagenumbering{arabic}

\section{Introduction}

The Inverted pendulum is a classical control problem used for illustrating non-linear control techniques. The system is motivated by applications such as control of rockets and anti-seismic control of buildings.The inverted pendulum system consists of a pendulum which is attached to a cart that can move horizontally. It is unstable because its center of mass is above its pivot point so without additional help will fall over.The cart move horizontally by an external. This force is the only control input to the system. By manipulating the force, the position of pendulum is to be controlled.\\
Considering the force, we would like to simulate the behaviour of the system and measure the position and the angle of the pendulum in the presence of the force.
The following section provides an overview of Software Requirement Specification (SRS) for a pole-cart system. This section explains the purpose of the document, the scope of requirements, the characteristic of the intended reader, and the organization of a document.

\subsection{Purpose of Document}

The main purpose of this document is to describe the mathematical model of IP system. To clarify software requirements, detailed information about goals, assumptions, theoretical models, and definitions will be explained.\\
Therefore, this document will be the reference during producing the software. It will provide  requirements of the software which will be used in
planing for design stage.
The following SRS document will remain abstract exploring what is being solved rather than how it will be solved.

\subsection{Scope of Requirements} 

The problem is restricted to two dimensions. The pendulum is attached to the cart in such a way that it is free to rotate in the pivot. The cart can move horizontally,left or right. for more details, see the assumptions section  (Section~\ref{sec_assumpt})
 
\subsection{Characteristics of Intended Reader} \label{sec_IntendedReader}
Intended Reader(s) or Reviewer(s) should have knowledge about equations of motion in Physics. Good understanding of solving an ordinary differential equation (ODE) in Mathematics is also needed. Both of the topics are offered in the undergrad engineering courses or high school Mathematics and Physics subjects.


\subsection{Organization of Document}

The organization of the document follows the template for an SRS for scientific computing software proposed by \cite{article1} The template will present the system's goals, theories, definitions, and assumptions. Goals are good points to start. To be familiar with mathematical model, Section~\ref{sec_theoretical} and then Section~\ref{sec_theoretical}, Section~\ref{sec_gendef}, Section~\ref{sec_instance} can be explored.
Readers interested in top-down reading can begin by reading the system's goal statements (Section~\ref{Sec_gs}). Subsequently, the theoretical models will elaborate on the goal statements. Lastly, readers can finish with a more  understanding of the system by reading instance models of the system.

\section{General System Description}

This section provides general information about the system.  It identifies the
interfaces between the system and its environment, describes the user
characteristics and lists the system constraints.  

\subsection{System Context}


\begin{figure}[h!]
\begin{center}
\includegraphics[width=0.6\textwidth]{systemcontext.jpg}
\caption{System Context}
\label{Fig_SystemContext} 
\end{center}
\end{figure}

The interaction between the product and the user is through a user interface. The responsibilities of the user and the system are as follows:
\begin{itemize}
\item User Responsibilities:
\begin{itemize}
\item Provide required inputs including cart and pole features.
\end{itemize}
\item The Inverted Pendulum simulation software Responsibilities:
\begin{itemize}
\item Ensure all inputs are in correct format and detect data type mismatch.
\item Calculate and show the outputs.
\end{itemize}
\end{itemize}

\subsection{User Characteristics} \label{SecUserCharacteristics}

The end user of the Inverted Pendulum simulation software should have an understanding of undergraduate Calculus and Physics.

\subsection{System Constraints}

There are no system constraints for this project.

\section{Specific System Description}

This section first presents the problem description, which gives a high-level view of the problem to be solved.  This is followed by the solution characteristics
specification, which presents the assumptions, theories, definitions and finally the instance models.

\subsection{Problem Description} \label{Sec_pd}
The inverted pendulum system inherently has two equilibrium points, one of which is stable while the other is unstable. The stable equilibrium corresponds to a state in which the pendulum is pointing downwards. In the absence of any
control force, the system will naturally return to this state. The unstable equilibrium corresponds to a state in which the pendulum points strictly upwards and thus requires a control force to maintain this position.\\

The Inverted Pendulum simulation tool is intended to describe and simulate the behaviour of the system in the presence of different inputs. As the inverted pendulum is widely used as a benchmark to demonstrate concepts and test control strategies and algorithms in nonlinear control, the final tool can show system status and be a great help as an educational material for students and engineers.


\subsubsection{Terminology and  Definitions}

This subsection provides a list of terms that are used in the subsequent
sections and their meaning, with the purpose of reducing ambiguity and making it easier to correctly understand the requirements:

\begin{itemize}
\item{Inverted Pendulum: An inverted pendulum is a pendulum that has its center of mass above its pivot point. It is unstable and without additional help will fall over.}

\item {Force equilibrium: A body is in force equilibrium if the sum of all the forces acting on the body is zero.}

\item{Friction: The force that resists the sliding or rolling of one solid object over another.}

\item{ Moment: The turning effect of a force is called the moment. The moment is the result of the force multiplied by the perpendicular distance from the line of action of the force to the pivot or point where the object will turn.}
\end{itemize}

\subsubsection{Physical System Description} \label{sec_phySystDescrip}
  
The physical system of IP simulation Tool, as shown in Figure~\ref{fig_physys},
includes the following elements:

\begin{itemize}

\item{PS1: The cart}
\item{PS2: The pendulum}

\end{itemize}

%\plt{A figure here makes sense for most SRS documents}

 \begin{figure}[h!]
\begin{center}
 %\rotatebox{-90}
 
\includegraphics[width=0.5\textwidth]{physic_system.jpg}
 \caption{The physical system of IP simulation Tool}
 \label{fig_physys}
 \end{center}
 \end{figure}
 
\subsubsection{Goal Statements}
 Given the cart and pendulum features and external force, the goal statements are:

\begin{itemize}
\item[GS\refstepcounter{goalnum}\thegoalnum \label{G_model}:]
Present a mathematical model of IP
\item[GS\refstepcounter{goalnum}\thegoalnum \label{G_calPosition}:]
Calculate the position of the cart
\item[GS\refstepcounter{goalnum}\thegoalnum \label{G_calTheta}:]
Calculate the angle of the pendulum arm
\end{itemize}

\subsection{Solution Characteristics Specification}


\subsubsection{Assumptions} \label{sec_assumpt}

This section simplifies the original problem and helps in developing the
theoretical model by filling in the missing information for the physical
system. The numbers given in the square brackets refer to the theoretical model
[T], general definition [GD], data definition [DD], instance model [IM], or
likely change [LC], in which the respective assumption is used.

\begin{itemize}

\item[A\refstepcounter{assumpnum}\theassumpnum \label{A_systemdimension}:]
The system is a two-degree of freedom system, so the pendulum is allowed to rotate in a two-dimensional plane.
\item[A\refstepcounter{assumpnum}\theassumpnum \label{A_pendfrictionl}:]
The pendulum is smooth and friction-less.
\item[A\refstepcounter{assumpnum}\theassumpnum \label{A_pendulumInertia}:]
The pendulum is unified and its moment of inertia is constant.
\item[A\refstepcounter{assumpnum}\theassumpnum \label{A_cosTheta}:]
The equations of the problem are non linear equations and to make them linear, we can assume that $\theta$ is small enough to have, $\cos(\theta)$$ is equal to $1.
\item[A\refstepcounter{assumpnum}\theassumpnum \label{A_sinTheta}:]
According to previous assumption, for small angle of $\theta$, $\sin(\theta)$ is equal to $\theta$.\\
\end{itemize}


\subsubsection{Theoretical Models}\label{sec_theoretical}

This section focuses on the general equations and laws that the tool is based
on. 

~\newline
\noindent
\begin{minipage}{\textwidth}
\renewcommand*{\arraystretch}{1.5}
\begin{tabular}{| p{\colAwidth} | p{\colBwidth}|}
\hline
\rowcolor[gray]{0.9}
Number& TM\refstepcounter{theorynum}\thetheorynum\label{NSL}\\
\hline
Label&  \bf Newton's Second Law of Motion\\
\hline
Equation & {$ F = m.a$ }\\ 
\hline
Description & {Newton�s second law of motion pertains to the behavior of objects for which all existing forces are not balanced. The second law states that the acceleration of an object is dependent upon two variables, the net force acting upon the object which is $a$ and the mass of the object that is $m$. The acceleration of an object depends directly upon the net force acting upon the object, and inversely upon the mass of the object. As the force acting upon an object is increased, the acceleration of the object is increased. As the mass of an object is increased, the acceleration of the object is decreased.}  \\
\hline
Notes & none. \\ 
\hline
Sources& \url{https://www.physicsclassroom.com/class/newtlaws/Lesson-3/Newton-s-Second-Law} \\
\hline
Ref.\ By &  \dref{NLHC},\dref{NLHP},\dref{NLVP}, \iref{Findx} \\
\hline
\end{tabular}
\end{minipage}\\


~\newline
\noindent
\begin{minipage}{\textwidth}
\renewcommand*{\arraystretch}{1.5}
\begin{tabular}{| p{\colAwidth} | p{\colBwidth}|}
\hline
\rowcolor[gray]{0.9}
Number& TM\refstepcounter{theorynum}\thetheorynum\label{NSLR}\\
\hline
Label&  \bf Newton�s Second Law for Rotation\\
\hline
Equation & {$\sum(\tau)$=$I$.$\alpha$ }\\ 
\hline
Description & {If more than one torque acts on a rigid body about a fixed axis, then the sum of the torques equals the moment of inertia times the angular acceleration.}  \\
&  $I$ is the moment of inertia for a point particle. \\
& $\alpha$ is angular acceleration of the object.\\
\hline
Notes & none. \\ 
\hline
Sources& \url{https://www.physicsclassroom.com/class/newtlaws/Lesson-3/Newton-s-Second-Law} \\
\hline
Ref.\ By &  \iref{FindTehta} \\
\hline
\end{tabular}
\end{minipage}\\


~\newline
\noindent
\begin{minipage}{\textwidth}
\renewcommand*{\arraystretch}{1.5}
\begin{tabular}{| p{\colAwidth} | p{\colBwidth}|}
\hline
\rowcolor[gray]{0.9}
Number& TM\refstepcounter{theorynum}\thetheorynum\label{TorqueDef}\\
\hline
Label&  \bf Torque Definition \\
\hline
Equation & {$\tau=rF\sin(\theta)$ }\\ 
\hline
Description & {Measure of the twisting action caused by a force that can cause an object to rotate about an axis.}  \\
& $\tau$ is torque \\
& $F$ is applied force\\
&$r$ is the radius from the axis of rotation to the location where the force is exerted\\
&$\theta$ is the angle between $F$ and $r$\\
\hline
Notes & none. \\ 
\hline
Sources& \url{https://www.khanacademy.org/science/in-in-class11th-physics/in-in-system-of-particles-and-rotational-motion/in-in-torque-and-equilibrium-ap/a/torque-and-equilibrium} \\
\hline
Ref.\ By & \dref{AngDis}  \\
\hline
\end{tabular}
\end{minipage}\\

 
 ~\newline
 \noindent
 \begin{minipage}{\textwidth}
 \renewcommand*{\arraystretch}{1.5}
 \begin{tabular}{| p{\colAwidth} | p{\colBwidth}|}
 \hline
 \rowcolor[gray]{0.9}
 
Number& TM\refstepcounter{theorynum}\thetheorynum \label{FoG}\\
\hline
Label&  \bf Force of Gravity\\
\hline
Equation & {$ W = m.g$ } \\ 
\hline
Description & { Gravity is a force that attracts objects toward the Earth. It is an approximation of the gravitational force that attracts objects of mass toward each other at great distances.}\\
&$g$ is the acceleration due to gravity\\ 
&$m$ is the mass of the object \\ 
\hline
Notes & none. \\
\hline
Sources& \url{https://www.school-for-champions.com/science/gravity_overview.htm#.Y9-9ZXbMLs0} \\
\hline
Ref.\ By &  \dref{NLVP}\\
\hline
\end{tabular}
\end{minipage}\\



\subsubsection{General Definitions}\label{sec_gendef}

This section collects the laws and equations that will be used in building the
instance models.
~\newline

\noindent
\begin{minipage}{\textwidth}
\renewcommand*{\arraystretch}{1.5}
\begin{tabular}{| p{\colAwidth} | p{\colBwidth}|}
\hline
\rowcolor[gray]{0.9}
Number& GD\refstepcounter{defnum}\thedefnum \label{NLHC}\\
\hline
Label &\bf Newton's second law of motion for cart in horizontal \\
\hline
% Units&$MLt^{-3}T^0$\\
% \hline
SI Units& \si{\newton} \\
\hline
Equation& $M_c\ddot{x}= F-H_c-b\dot{x}$,  \\
\hline
Description & Based on Newton's second law of motion, sum of all forces applied to the cart is equal to $M_c.a$.
\\
& $F$ is the external force exerted to the cart in \si{\newton}.\\
& $H_c$ is the force component in {x} direction of the reaction force which pendulum applies on the cart and measured in \si{\newton}.\\
& $b$ is coefficient of friction for  the cart.\\
& $\dot{x}$ is the velocity of cart and therefore $\ddot{x}$ will be its acceleration.\\
\hline
  Source & \url{https://ctms.engin.umich.edu/CTMS/index.php?example=InvertedPendulum&section=SystemModeling} \\
  \hline
  Ref.\ By & \iref{Findx}\\
  \hline

\end{tabular}
\end{minipage}\\

~\newline

\noindent
\begin{minipage}{\textwidth}
\renewcommand*{\arraystretch}{1.5}
\begin{tabular}{| p{\colAwidth} | p{\colBwidth}|}
\hline
\rowcolor[gray]{0.9}
Number& GD\refstepcounter{defnum}\thedefnum \label{NLHP}\\
\hline
Label &\bf Newton's second law of motion for pendulum in horizontal \\
\hline
% Units&$MLt^{-3}T^0$\\
% \hline
SI Units& All forces are measured in \si{\newton} \\
\hline
Equation& $m_p\ddot{x}= H_p$ \\
\hline
Description &Based on Newton's second law of motion, sum of all forces applied to the pendulum in {$x$} direction is equal to $m_p.a$.
\\

& $H_p$ is the force component in {$x$} direction of the force exerted by cart on the pendulum and measured in \si{\newton}.\\
& $m_p$ is mass of pendulum.\\
& $\ddot{x}$ is the pendulum acceleration.\\
\hline
  Source & \url{https://ctms.engin.umich.edu/CTMS/index.php?example=InvertedPendulum&section=SystemModeling }\\
  \hline
  Ref.\ By & \iref{FindTehta}\\
  \hline
  \end{tabular}
\end{minipage}\\

~\newline

\noindent
\begin{minipage}{\textwidth}
\renewcommand*{\arraystretch}{1.5}
\begin{tabular}{| p{\colAwidth} | p{\colBwidth}|}
\hline
\rowcolor[gray]{0.9}
Number& GD\refstepcounter{defnum}\thedefnum \label{NLVP}\\
\hline
Label &\bf Newton's second law of motion for pendulum in vertical \\
\hline
% Units&$MLt^{-3}T^0$\\
% \hline
SI Units& All forces are measured in \si{\newton} \\
\hline
Equation& $m_p\ddot{(l-l\cos(\theta))}= (m_p)g-V_p$ \\
\hline
Description & Based on Newton's second law of motion, sum of all forces applied to the pendulum in {$y$} direction is equal to $m_p.a$. Because the second derivation of $l$ as a constant values is zero, therefore the equation reduces to $m_p\ddot{(-l\cos(\theta))}= (m_p)g-V_p$.\\

& $V_p$ is the force component in {$y$} direction of the force exerted by cart on the pendulum and measured in \si{\newton}.\\
& $m_p$ is mass of pendulum.\\
& $l-l\cos(\theta)$ is the Vertical displacement of the pendulum.\\
\hline
Source &\url{https://ctms.engin.umich.edu/CTMS/index.php/example=InvertedPendulum&section=SystemModeling }\\
  \hline
  Ref.\ By & \iref{FindTehta}\\
  \hline
  \end{tabular}
\end{minipage}\\



\noindent
\begin{minipage}{\textwidth}
\renewcommand*{\arraystretch}{1.5}
\begin{tabular}{| p{\colAwidth} | p{\colBwidth}|}
\hline
\rowcolor[gray]{0.9}
Number& GD\refstepcounter{defnum}\thedefnum \label{AngDis}\\
\hline
Label &\bf Calculate angular displacement of the pendulum \\
\hline
% Units&$MLt^{-3}T^0$\\
% \hline
SI Units& \tau\\
\hline
Equation& $I$.$\ddot{\theta}$ =$l$.( $V.\sin(\theta)$ - $H.\cos(\theta)$)\\
\hline
Description & Let's take the center of mass of this pendulum as a center of rotation so this pendulum rotates about this center of mass. The torques due to forces exerted to pendulum will calculate as below: the torque of weight force of pendulum is zero since the perpendicular distance of this force uh from center of rotation is zero. For reaction forces $Vp$, $Hp$, we decompose these forces to components which are perpendicular to the pendulum which we have, $V.\sin(\theta)$ and $H.\cos(\theta)$.\\
& $I$ is the moment of inertia of the pendulum.\\
& $\ddot{\theta}$ is pendulum angular acceleration.\\
\hline
  Source & \url{https://ctms.engin.umich.edu/CTMS/index.php?example=InvertedPendulum&section=SystemModeling} \\
  \hline
  Ref.\ By & \iref{FindTehta}\\
  \hline

\end{tabular}
\end{minipage}\\


%\subsubsection*{Detailed derivation of the member length}

%\plt{This may be necessary when the necessary information does not fit in the description field.}
%\plt{Derivations are important for justifying a given GD.  You want it to be clear where the equation came from.}

\subsubsection{Data Definitions}\label{sec_datadef}

%\plt{The Data Definitions are definitions of symbols and equations that are given for the problem.  They are not derived; they are simply used by other models.  For instance, if a problem depends on density, there may be a data  definition for the equation defining density.  The DDs are given information that you can use in your other modules.}

%\plt{All Data Definitions should be used (referenced) by at least one other model.}

This section collects and defines all the data needed to build the instance models. The dimension of each quantity is also given.  %\plt{Modify the examples below for your problem, and add additional definitions as appropriate.}

~\newline

\noindent
\begin{minipage}{\textwidth}
\renewcommand*{\arraystretch}{1.5}
\begin{tabular}{| p{\colAwidth} | p{\colBwidth}|}
\hline
\rowcolor[gray]{0.9}
Number& DD\refstepcounter{datadefnum}\thedatadefnum \label{HDC}\\
\hline
Label& \bf 
Horizontal displacement of the pendulum\\
\hline
Symbol &$d_px$\\
\hline

  SI Units & \si{\metre}\\
  \hline
  Equation&$ d_px =l\sin(\theta) + {x(t)} $\\
 \hline
Description & 
The distance traveled by the center of mass of the pendulum is due to two reasons, first, horizontal displacement of the cart, and second, due to angular rotation of the pendulum. The distance due to angular rotation in the pendulum will be $ =l\sin(\theta)$ while the half length of the pendulum is $l$, and the angular rotation in the pendulum is $\theta(t)$.
Therefore, the total distance traveled by the center of mass of pendulum is equal to $ =l\sin(\theta) + {x(t)} $.\\
 \hline
  Sources & \url{https://www.youtube.com/watch?v=c3z4eo6s0Ek} \\
  \hline
  Ref.\ By & \dref{NLHC}\\
  \hline
\end{tabular}
\end{minipage}\\
%------------------------------------DD2
~\newline



\noindent
\begin{minipage}{\textwidth}
\renewcommand*{\arraystretch}{1.5}
\begin{tabular}{| p{\colAwidth} | p{\colBwidth}|}
\hline
\rowcolor[gray]{0.9}
Number& DD\refstepcounter{datadefnum}\thedatadefnum \label{VDP}\\
\hline
Label& \bf  Vertical displacement of the pendulum\\
\hline

Symbol &$d_py$\\
\hline

  SI Units & metre\\
  \hline
  Equation& $d_py= l-l\cos(\theta)$\\
 \hline
Description & When the pendulum rotates, the center of mass of the pendulum travels a vertical displacement, and as the half length of the pendulum is $l$, so the vertical distance travelled by the center of mass of the pendulum will be equal to $l\cos(\theta)$ hence the total displacement is equal to $l-l\cos(\theta)$
   \\
  \hline
  Sources & \url{https://www.youtube.com/watch?v=c3z4eo6s0Ek} \\
  \hline
  Ref.\ By & \dref{NLHP}\\
  \hline
\end{tabular}
\end{minipage}\\




\noindent
\begin{minipage}{\textwidth}
\renewcommand*{\arraystretch}{1.5}
\begin{tabular}{| p{\colAwidth} | p{\colBwidth}|}
\hline
\rowcolor[gray]{0.9}
Number& DD\refstepcounter{datadefnum}\thedatadefnum \label{Inertia of pendulum}\\
\hline
Label& \bf The moment of inertia of the pendulum\\
\hline

Symbol &$I$\\
\hline

  SI Units & \si{\kilogram\metre^2}\\
  \hline
  
  Equation& $I$= $\frac{1}{12}($m_p.l$)$ \\
  
 \hline
Description & The moment of inertia of a rod is related to its mass and length.   \\
& the mass of rod is denoted by $m$ \\
& and its length is denoted by $l$.\\ 
  \hline
  Sources & \url{https://www.vedantu.com/question-answer/moment-of-a-rod-of-mass-m-length-l-about-class-12-physics-cbse-5f471db855e8473a85a108c5} \\
  \hline
  Ref.\ By & \dref{AngDis}\\
  %accrding to assumption3
  \hline
\end{tabular}
\end{minipage}\\



\subsubsection{Data Types}\label{sec_datatypes}

This section collects and defines all the data types needed to document the
models.

~\newline

\noindent
\begin{minipage}{\textwidth}
\renewcommand*{\arraystretch}{1.5}
\begin{tabular}{| p{\colAwidth} | p{\colBwidth}|}
  \hline
  \rowcolor[gray]{0.9}
  Type Name & Name for Type\\
  \hline
  Type Def & \\
  \hline
  Description & 
  \\
  \hline
  Sources & \\
  \hline
\end{tabular}
\end{minipage}\\

\subsubsection{Instance Models} \label{sec_instance}    

~\newline

%Instance Model 1

\noindent
\begin{minipage}{\textwidth}
\renewcommand*{\arraystretch}{1.5}
\begin{tabular}{| p{\colAwidth} | p{\colBwidth}|}
  \hline
  \rowcolor[gray]{0.9}
  Number& IM\refstepcounter{instnum}\theinstnum \label{Findx}\\
  \hline
  Label& \bf Equation of motion for the cart\\
  \hline
  Input& $F$, $M_c$, $m_p$, $b$, $l$ from \iref{}\\0
  & The input is constrained so that $\theta$ is small\\
  \hline
  Output& $x(t)$, such that $F$= $ (M_c+m_p)\ddot{x}+ b\dot{x} + m_pl\ddot{\theta}\cos{\theta} - m_pl(\dot{\theta}) ^ 2 \sin{\theta} \rightarrow$\\
&$\ddot{x} =\dfrac{F- b\dot{x}- m_pl\ddot{\theta}\cos{\theta}+ m_pl(\dot{\theta}) ^ 2 \sin{\theta}}{(M_c+m_p)}$ \\
 
  \hline
  Description&
  $M_c$ is the mass of the cart (\si{\kilogram}).\\
  &$m_p$ is the mass of the pendulum (\si{\kilogram}).\\
  &$b$ is the coefficient friction of the cart (dimensionless).\\
  &$l$ is the length of pendulum.\\
  %&$\eta = \frac{h_P A_P}{h_C A_C}$ is a constant (dimensionless).\\
  %& The above equation applies as long as the water is in liquid form,
  %$0<T_W<100^o\text{C}$, where $0^o\text{C}$ and $100^o\text{C}$ are the melting
  %and boiling points of water, respectively (\aref{A_OpRange}, \aref{A_Pressure}).
  
  \hline
  Sources& \url{https://www.mathworks.com/help/control/ug/control-of-an-inverted-pendulum-on-a-cart.html }\\
  \hline
  Ref.\ By & \\
  \hline
\end{tabular}
\end{minipage}\\

%~\newline
%Instance Model 1

\noindent
\begin{minipage}{\textwidth}
\renewcommand*{\arraystretch}{1.5}
\begin{tabular}{| p{\colAwidth} | p{\colBwidth}|}
  \hline
  \rowcolor[gray]{0.9}
  Number& IM\refstepcounter{instnum}\theinstnum \label{FindTehta}\\
  \hline
  Label& \bf Equation of motion for the pendulum\\
  \hline
  Input&  $M_c$, $m_p$, $b$, $l$\\

  \hline
  Output & $\theta(t)$, such that $(I+m_pl^2)\ddot{\theta} + m_pl\ddot{x}\cos{\theta} - m_pgl\sin{\theta} = 0 $\rightarrow\\
  &$\ddot{\theta}$= $\dfrac{m_pgl\sin{\theta}-m_pl\ddot{x}\cos{\theta}}{(I+m_pl^2)}$\\
  & \\
  \hline
  Description&
  $M_c$ is the mass of the cart (\si{\kilogram}).\\
  &$m_p$ is the mass of the pendulum (\si{\kilogram}).\\
  &$l$ is the length of pendulum.  \\
  \hline
  Sources&  \url{https://www.mathworks.com/help/control/ug/control-of-an-inverted-pendulum-on-a-cart.html } \\
  \hline
  Ref.\ By &\\
  \hline
\end{tabular}
\end{minipage}\\

%~\newline

%\subsubsection*{Derivation of ...}

%\plt{The derivation shows how the IM is derived from the TMs/GDs.  In cases
 % where the derivation cannot be described under the Description field, it will
  %be necessary to include this subsection.}

\subsubsection{Input Data Constraints} \label{sec_DataConstraints}    

Table~\ref{TblInputVar} shows the data constraints on the input output
variables.  The column for physical constraints gives the physical limitations
on the range of values that can be taken by the variable.  The column for
software constraints restricts the range of inputs to reasonable values.  The
software constraints will be helpful in the design stage for picking suitable
algorithms.  The constraints are conservative, to give the user of the model the
flexibility to experiment with unusual situations.  The column of typical values
is intended to provide a feel for a common scenario.  The uncertainty column
provides an estimate of the confidence with which the physical quantities can be
measured.  This information would be part of the input if one were performing an
uncertainty quantification exercise.

The specification parameters in Table~\ref{TblInputVar} are listed in
Table~\ref{TblSpecParams}.

\begin{table}[!h]
  \caption{Input Variables} \label{TblInputVar}
  \renewcommand{\arraystretch}{1.2}
\noindent \begin{longtable*}{l l l l c} 
  \toprule
  \textbf{Var} & \textbf{Physical Constraints} & \textbf{Software Constraints} &
                             \textbf{Typical Value} & \textbf{Uncertainty}\\
  \midrule 
  $m_p$ & $m_p > 0$ & $m_p > 0$ & 0.2 \si[per-mode=symbol] {\kilogram} &  \\
 $M_c$ & $M_c > 0$ & $M_c > 0$ & 0.5 \si[per-mode=symbol] {\kilogram} &  \\
 $b$ & $b > 0$ & $b > 0$ & 0.1\si[per-mode=symbol]  (dimensionless) &  \\
 $l$ & $l > 0$ & $l> 0$ & 0.3 \si[per-mode=symbol] {\metre} &  \\
 $I$ & $I > 0$ & $I> 0$ & 0.006 \si[per-mode=symbol] {\kilogram. \metre^2} &  \\


  \bottomrule
\end{longtable*}
\end{table}

\begin{table}[!h]
\caption{Specification Parameter Values} \label{TblSpecParams}
\renewcommand{\arraystretch}{1.2}
\noindent \begin{longtable*}{l l} 
  \toprule
  \textbf{Var} & \textbf{Value} \\
  \midrule 
   & \\
  \bottomrule
\end{longtable*}
\end{table}

\subsubsection{Properties of a Correct Solution} \label{sec_CorrectSolution}

\noindent
Table~\ref{TblOutputVar}shows the physical constraints on the output.

\begin{table}[!h]
\caption{Output Variables} \label{TblOutputVar}
\renewcommand{\arraystretch}{1.2}
\noindent \begin{longtable*}{l l} 
  \toprule
  \textbf{Var} & \textbf{Physical Constraints} \\
  \midrule 
  $\theta$ &  $0\leq \theta \leq 360$\\
 $x$ &  $ 0\leq x$ \\
  \bottomrule
\end{longtable*}
\end{table}

\section{Requirements}

This section provides the functional requirements, and the business tasks that the
IP simulation software is expected to complete, and the nonfunctional requirements, the qualities that the software is expected to exhibit.

\subsection{Functional Requirements}

\noindent \begin{itemize}

\item[R\refstepcounter{reqnum}\thereqnum \label{R_Inputs}:] 
{Provide input values.}

\item[R\refstepcounter{reqnum}\thereqnum \label{R_OutputInputs}:] {Calculate horizontal position of the cart form \iref{Findx}}

\item[R\refstepcounter{reqnum}\thereqnum \label{R_Calculate}:] {Calculate the angular position of the pendulum from \iref{FindTehta}}

\end{itemize}

\subsection{Nonfunctional Requirements}
The nonfunctional requirements are listed below:
\noindent \begin{itemize}

\item[NFR\refstepcounter{nfrnum}\thenfrnum \label{NFR_Accuracy}:]
  {Accuracy: The accuracy of the computed solutions should meet the level needed for engineering or scientific application and have the properties described in Section \ref{sec_CorrectSolution}.}
 
\item[NFR\refstepcounter{nfrnum}\thenfrnum \label{NFR_Usability}:] 
{Usability: The properties of the software should be able to tested easily through verification and validation plan (VnV Plan).}
 

\item[NFR\refstepcounter{nfrnum}\thenfrnum \label{NFR_Maintainability}:]
{Maintainability: The effort required to make any of the likely    changes listed for IP simulation software should be less than 50\% of the original development time.}

\item[NFR\refstepcounter{nfrnum}\thenfrnum \label{NFR_Portability}:]
{Portability:  IP simulation software can be run on all of the
    possible operating environments, such as Windows, Mac-OS, and Linux.} 
\end{itemize}

\section{Likely Changes}    

\noindent \begin{itemize}

\item[LC\refstepcounter{lcnum}\thelcnum\label{LC_meaningfulLabel}:] 
{The software may be changed to consider friction of the pendulum [\aref{A_pendfrictionl} ]}

\item[LC\refstepcounter{lcnum}\thelcnum\label{LC_meaningfulLabel}:] 
{The software may be changed to consider non- linear equations and all values of the angle ot the pendulum [\aref{A_cosTheta}],[\aref{A_sinTheta}]}


\end{itemize}

\section{Unlikely Changes}    

\noindent \begin{itemize}

\item[ULC1\label{LC_meaningfulLabel}:]
{The system has two-degree of freedom [\aref{A_systemdimension}].}

\end{itemize}

\section{Traceability Matrices and Graphs}

The purpose of the traceability matrices is to provide easy references on what
has to be additionally modified if a certain component is changed.  Every time a
component is changed, the items in the column of that component that are marked
with an ``X'' may have to be modified as well.  Table~\ref{Table:trace} shows the
dependencies of theoretical models, general definitions, data definitions, and
instance models with each other.Table~\ref{Table:A_trace} shows the dependencies of theoretical models, general definitions, data definitions, instance models, and likely changes on the assumptions.Table~\ref{Table:R_trace} shows the
dependencies of instance models, requirements, and data constraints on each
other. 


%\afterpage{
\begin{landscape}
\begin{table}[h!]
\centering
\begin{tabular}{|c|c|c|c|c|c|c|c|c|c|c|c|c|c|}
\hline
	& \tref{NSL}&\tref{NSLR}&\tref{TorqueDef}&\tref{FoG}& \dref{NLHC}& \dref{NLHP}& \dref{NLVP}& \dref{AngDis}& \ddref{HDC}&\ddref{VDP}&\ddref{Inertia of pendulum} &\iref{Findx}&\iref{FindTehta}\\
\hline
\tref{NSL}    & & & & &X &X &X & & & & &X &\\ \hline
\tref{NSLR}    & & & & & & & & & & & & &X\\ \hline
\tref{TorqueDef}    & & & & & & & & & & &X & &\\ \hline
\tref{FoG}    & & & & & & & & X& & & & &\\ \hline
\dref{NLHC}  & & & & & & & & & & & &X & \\ \hline
\dref{NLHP}  & & & & & & & & & & & & &X\\ \hline
\dref{NLVP}  & & & & & & & & & & & & &X\\ \hline
\dref{AngDis}& & & & & X& & & & & & & & \\ \hline
\ddref{HDC}  & & & & & & X& & & & & & & \\ \hline
\ddref{VDP}  & & & & & & X& & & && & & \\ \hline
\ddref{Inertia of pendulum}       & & & & & & X& & & & & & &\\ \hline
\iref{Findx}  & & & & & & & & & & & & &\\ \hline
\iref{FindTehta} & & & & & & & & & & & & &\\ \hline
\hline
\end{tabular}
\caption{Traceability Matrix Showing the Connections Between Items of Different SectionsAssumptions and Other Items}
\label{Table:trace}
\end{table}
\end{landscape}
%}




\begin{table}[h!]
\centering
\begin{tabular}{|c|c|c|c|c|c|c|c|c|c|c|c|c|c|}
\hline        
	& \tref{NSL}&\tref{NSLR}&\tref{TorqueDef}&\tref{FoG}& \dref{NLHC}& \dref{NLHP}& \dref{NLVP}& \dref{AngDis}& \ddref{HDC}&\ddref{VDP}&\ddref{Inertia of pendulum} &\iref{Findx}&\iref{FindTehta}\\
\hline
\aref{A_systemdimension}      & & & & & & & & & & & &X &X\\ \hline
\aref{A_pendfrictionl}        & & & & & &X &X & & & & &X &\\ \hline
\aref{A_pendulumInertia}      & & & & & & & &X & & & & & X\\ \hline
\aref{A_cosTheta}             & & & & & & & & & & & & X& X\\ \hline
\aref{A_sinTheta}             & & & & & & & & & & & & X& X\\ \hline

\hline
\end{tabular}
\caption{Traceability Matrix Showing the Connections Between Assumptions and Other Items}
\label{Table:A_trace}
\end{table}

\begin{table}[h!]
\centering
\begin{tabular}{|c|c|c|c|}
\hline
	& \iref{Findx}&\iref{FindTehta}&\ref{sec_DataConstraints} \\
\hline
\rref{R_Inputs}     & & &X\\ \hline
\rref{R_OutputInputs}    &X &X &\\ \hline
\rref{R_Calculate}   &X &X &\\ \hline

\end{tabular}
\caption{Traceability Matrix Showing the Connections Between Requirements and Instance Models}
\label{Table:R_trace}
\end{table}


\section{Development Plan}

\section{Values of Auxiliary Constants}

\plt{Show the values of the symbolic parameters introduced in the report.}

\plt{The definition of the requirements will likely call for SYMBOLIC\_CONSTANTS.
Their values are defined in this section for easy maintenance.}

\plt{The value of FRACTION, for the Maintainability NFR would be given here.}

\newpage

\bibliographystyle {plainnat}
\bibliography {ref}

\newpage
\section*{Appendix --- Reflection}

The information in this section will be used to evaluate the team members on the
graduate attribute of Lifelong Learning.  Please answer the following questions:

\begin{enumerate}
  \item What knowledge and skills will the team collectively need to acquire to
  successfully complete this capstone project?  Examples of possible knowledge
  to acquire include domain specific knowledge from the domain of your
  application, or software engineering knowledge, mechatronics knowledge or
  computer science knowledge.  Skills may be related to technology, or writing,
  or presentation, or team management, etc.  You should look to identify at
  least one item for each team member.
  \item For each of the knowledge areas and skills identified in the previous
  question, what are at least two approaches to acquiring the knowledge or
  mastering the skill?  Of the identified approaches, which will each team
  member pursue, and why did they make this choice?
\end{enumerate}

\end{document}