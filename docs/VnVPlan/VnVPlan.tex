\documentclass[12pt, titlepage]{article}
\usepackage{siunitx}
\usepackage{booktabs}
\usepackage{tabularx}
\usepackage{hyperref}
\hypersetup{
    colorlinks,
    citecolor=blue,
    filecolor=black,
    linkcolor=red,
    urlcolor=blue
}
\usepackage[numbers]{natbib}


\input{}
\newcommand{\colZwidth}{1.0\textwidth}
\newcommand{\colAwidth}{0.13\textwidth}
\newcommand{\colBwidth}{0.82\textwidth}
\newcommand{\colCwidth}{0.1\textwidth}
\newcommand{\colDwidth}{0.05\textwidth}
\newcommand{\colEwidth}{0.8\textwidth}
\newcommand{\colFwidth}{0.17\textwidth}
\newcommand{\colGwidth}{0.5\textwidth}
\newcommand{\colHwidth}{0.28\textwidth}


\begin{document}

\title{System Verification and Validation Plan for \\IP Simulator}
\author{Mina Mahdipour}
\date{\today}
	
\maketitle

\pagenumbering{roman}

\section{Revision History}

\begin{tabularx}{\textwidth}{p{4cm}p{2cm}X}
\toprule {\bf Date} & {\bf Version} & {\bf Notes}\\
\midrule
February 13, 2023 & 1.0 & Created the first version of VnV\\
February 13, 2023 & 1.1 & Added the plan part\\
February 17, 2023 & 1.2 & Added system test description\\
February 18, 2023 & 1.3 & Updated test section\\
February 20, 2023 & 1.5 & Updated NFRs test cases\\
April 15, 2023 & 2 & Updated all the documents\\
\bottomrule
\end{tabularx}

\newpage

\tableofcontents

\listoftables


\newpage

\section{Symbols, Abbreviations and Acronyms}

\renewcommand{\arraystretch}{1.2}
\begin{tabular}{l l} 
  \toprule		
  \textbf{symbol} & \textbf{description}\\
  \midrule 
 
  FR & Functional Requirements\\
  MIS & Module Interface Specification\\
  NFR & Nonfunctional Requirements\\
  N/A & Not Applicable\\
  VnV & Verification and Validation\\
SRS & Software Requirements Specification\\
 T & Test\\
 TC& Test Case\\
\bottomrule
\end{tabular}\\
\\
Section 1 of \href{https://github.com/MinMah23/CAS741-Project/tree/main/docs/SRS/SRS.pdf}{SRS} document can be referred by the reader for complete symbols used within the system.
\newpage

\pagenumbering{arabic}

This document outlines the roadmap of the verification and validation plan for the Inverted Pendulum (IP) Simulator to help ensure (but not prove) the correctness and completeness of the program.
It includes a plan for testing the functional and non-functional requirements, an overview of the system tests, and an outline of the unit tests.

\section{General Information}

\subsection{Summary}
This document provides the validation and verification plans for the IP Simulator Software. This software aims to simulate and describe the behavior of a cart-pendulum system in the presence of different values of external force. To do that, it will calculate the cart position and pendulum's angle after applying a time-varying force.
\subsection{Objectives}

The objectives of this document are listed below:
\begin{itemize}
\item Establish confidence in software reliability.
\item Demonstrate adequate usability.
\item Verify and validate the final product.
\end{itemize}
Performance testing, stress testing, and maintainability testing are outside of the scope of this document.
\subsection{Relevant Documentation}

\href{https://github.com/MinMah23/CAS741-Project/tree/main/docs/ProblemStatementAndGoals/ProblemStatement.pdf}{Problem Statement} presents an overview of the problem.
The requirements and an outline of the solution for the IP Simulator are captured in the \href{https://github.com/MinMah23/CAS741-Project/tree/main/docs/SRS/SRS.pdf}{Software Requirements Specification}, and verification of SRS is also a part of VnV process. The software design information will be captured in the \href{https://github.com/MinMah23/CAS741-Project/tree/main/docs/Design}{Module Interface Specification}. The final report of running the test which described in this document (unit test in section \ref{unittest}), will be shown in \href{https://github.com/MinMah23/CAS741-Project/tree/main/docs/VnVReport/VnVReport.pdf}{VnV Report}.

\section{Plan}
This section provides a roadmap of plans for different steps of producing the final product.
\subsection{Verification and Validation Team}
Our team details are described in table \ref{team}:

\begin{table}[h!]
\caption{Verification and Validation Team} \label{team}
\vspace*{3mm}
\begin{tabular}{| p{\colFwidth} | p{\colHwidth}|p{\colHwidth}|p{\colHwidth}|}
\hline
\textbf{Role} & \textbf{Responsibility} & \textbf{Assignee} & \textbf{Document} \label{R&R} \\
\hline
 Supervisor& Review and provide feedback & Dr.\ Spencer Smith& All documents\\
 \hline
Author& Create the products and refine them by feedback & Mina Mahdipour & All documents\\
\hline
 Domain Expert/ Reviewer& Review and provide feedback&Deesha Patel &All documents \\
  \hline
Secondary Reviewer&Review and provide feedback&Maryam Valian&SRS\\
\hline  
Secondary Reviewer&Review and provide feedback&Karen Wang&VnV plan\\
\hline 
Secondary Reviewer&Review and provide feedback&Joachim de Fourestier &MG and MIS\\
\hline
\end{tabular}
\end{table}



\subsection{SRS Verification Plan}

The SRS has been independently peer-reviewed by members of the VnV team according to \href{https://github.com/smiths/capTemplate/blob/main/docs/Checklists/SRS-Checklist.pdf}{SRS-Checklist}, consisting of the domain expert Deesha Patel and the SRS reviewer, Maryam Valian, as well as Dr.\ Spencer Smith.
The SRS document has been published to \href{https://github.com/MinMah23/CAS741-Project}{GitHub}. Any issues identified during the review were tracked and verified in Github platform.

\subsection{Design Verification Plan}

The design document MIS will be reviewed by the MIS review team, consisting of the domain expert, Deesha Patel and the MIS reviewer, Joachim de Fourestier, as well as Dr.\ Spencer Smith.
The MIS document will be published to GitHub. Defects will be addressed with issues on the GitHub platform. There is also \href{https://github.com/smiths/capTemplate/blob/main/docs/Checklists/MIS-Checklist.pdf}{MIS-Checklist}, that can be used.

\subsection{Verification and Validation Plan Verification Plan}
The VnV plan document will be reviewed by the VnV plan review team, consisting of the domain expert Deesha Patel and the reviewer, Karen Wang as well as Dr.\ Spencer Smith.
This document will be published to GitHub. Defects will be addressed with issues on the GitHub platform.
There is also \href{https://github.com/smiths/capTemplate/blob/main/docs/Checklists/VnV-Checklist.pdf}{VnV-Checklist}, that can be used.

\subsection{Implementation Verification Plan}

The IP Simulator will be developed in Python programming language. The implementation of the software will be tested under the test cases listed in section \ref{test}. The tests will either be automated or performed manually. \\
Static analysis will be done using the Code Walkthrough technique by the domain expert (Deesha Patel). During a meeting, the author presents the document under review, while the domain expert will run the test cases over the code by hand. She may also ask the author any relevant questions and discuss her findings with the author \cite{codeWalkThrough}. Visual Studio Code will be used for program debugging and for checking syntax errors. \\
 
\subsection{Automated Testing and Verification Tools}

Automated testing and verification tools will be extensively used in the development of IP Simulator for automation at different levels. These tools include \href{https://github.com/MinMah23/CAS741-Project}{git}, a distributed version-control system for tracking changes in source code during software development, \href{https://docs.pytest.org/en/7.2.x/}{Pytest} framework for unit, functional, and integration automated testing, and \href{https://flake8.pycqa.org/en/latest/}{Flake8} to check for errors, enforces coding standards, and identifies code complexity issues.

\subsection{Software Validation Plan}

There are no plans for the validation of the IP Simulator software.

\section{System Test Description}\label{test}
This section will define the tests to ensure IP Simulator meets the functional requirements seen in section 5 of the \href{https://github.com/MinMah23/CAS741-Project/tree/main/docs/SRS/SRS.pdf}{SRS} document for simulating inverted pendulum system. The subsections combine several requirements that are separated based on common ideas.

\subsection{Tests for Functional Requirements}
This section contains the system test cases for the functional requirements which are described in the  \href{https://github.com/MinMah23/CAS741-Project/tree/main/docs/SRS/SRS.pdf}{SRS}.

\subsubsection{Input validation \label{inp_val}}
This section includes tests to verify that the software rejects invalid input values such as non-numerical values or values outside of specified range. These tests also help to make sure that the IP Simulator accepts valid input values within specified range, FR1 and FR2 of requirements in section 5 of the  \href{https://github.com/MinMah23/CAS741-Project/tree/main/docs/SRS/SRS.pdf}{SRS}.

\begin{table}[h!]
\centering
\caption{Input Validation Test Cases} \label{tbl_inputValidation}
\vspace*{2mm}
 \begin{tabular}{|c|c c c c c| c c|} 
 \hline
   & &&\textbf{Input}&& &  &\textbf{Expected Output} \\ \hline
\textbf{TC} &   $m_p$ & $m_c$ & $l$ & $F$ & $b$  &   $Validity$   &   $Output$ \\ \hline
Valid-Inputs &0.3 & 0.4& 1.5& 20& 0.1 & Yes & t= 1 $x$= 6.4817 $\theta$= 0.5560\\
 \hline
I-Zero-Input-$m_p$ &0 & 0.5& 0.3& 50& 0.2 & No & Error: invalid input  \\
 \hline
I-Zero-Input-$m_c$ &0.2 & 0& 0.3& 50& 0.2& No & Error: invalid input\\
 \hline
I-Zero-Input-$l_p$ &0.2 & 0.5& 0& 50& 0.2 &  No & Error: invalid input\\
 \hline
V-Zero-Input-$b$ &0.2 & 0.5& 0.3& 50& 0 & Yes & t= 6 $x$=1023 $\theta$= 0.57\\
 \hline
V-Zero-Input-$f$ &0.2 & 0.5& 0.3& 0& 0.2& Yes & t= 1 $x$=0.3121 $\theta$= 0.8265\\
 \hline
 Neg-Input-$m_p$ &-0.2 & 0.5& 0.3& 50& 0.2&  No & Error: invalid input\\
 \hline
Neg-Input-$m_c$ &0.2 & -0.5& 0.3& 50& 0.2&   No & Error: invalid input\\
 \hline
 Neg-Input-$l_p$ &0.2 & 0.5& -0.3& 50& 0.2&   No & Error: invalid input\\
 \hline
 Neg-Input-$f$&0.2 & 0.5& 0.3& -50& 0.2 &   No & Error: invalid input\\
 \hline
 Neg-Input-$b$ &0.2 & 0.5& 0.3& 50& -0.2 &   No & Error: invalid input\\
 \hline
 OutofBound-$m_p$ & 51& 0.5& 0.3& 50& 0.2 & No & Error: invalid input\\
 \hline
 OutofBound-$m_c$ & 0.2& 51& 0.3& 50& 0.2 & No & Error: invalid input\\
 \hline
 OutofBound-$l_p$ & 0.2& 0.5& 11& 50& 0.2& No & Error: invalid input \\
 \hline
Null-Input-$m_p$ & & 0.5& 0.3& 50& 0.2 & No & Error: invalid input \\
  \hline
Null-Input-$m_c$ &0.2 & &0.3 & 50& 0.2 & No & Error: invalid input\\  
\hline
 Null-Input-$l_p$ & 0.2& 0.5& & 50& 0.2 & No & Error: invalid input \\
  \hline
  Null-Input-$f$ & 0.2& 0.5& 0.3& & 0.2 & No & Error: invalid input\\
  \hline
  Null-Input-$b$& 0.2& 0.5& 0.3& 50&  & No & Error: invalid input \\
  \hline

\end{tabular}

\end{table}		

 
\paragraph{Input Constraints Test}

\begin{enumerate}

\item{TC-Valid-Inputs: Valid Inputs}

Control: Automatic
					
Initial State: N/A 
					
Input: Set of input values as TC-Valid-Inputs test case in Table \ref{tbl_inputValidation}.

Output: $x$, the position of the cart equals  0.1410 and the angle of the pendulum, $\theta$, is 0.3643 when time, t, is 0.1 second. The initial values of $x_i$, $\dot{x_i}$, and $\dot{\theta{i}}$ are zero and $\theta{i}$ is 0.1.

Test Case Derivation: Derived from IP Simulator, as a normal and valid set of inputs.
					
How the test will be performed: Automatic test using Pytest.

\item{TC-I-Zero-Input: Zero Inputs}

Control: Automatic
					
Initial State: N/A 
					
Input: Set of invalid zero inputs as I-Zero-Input-$m_p$, I-Zero-Input-$m_c$, and I-Zero-Input-$l_p$ test cases in Table \ref{tbl_inputValidation}.
					
Output: Error message of invalid input because of zero inputs.

Test Case Derivation: Zero input for the mass of the pendulum or mass of the cart, or the length of the cart is not valid.
					
How the test will be performed: Automatic test using Pytest.

\item{TC-V-Zero-Input-$b$: Valid and Zero Inputs}

Control: Automatic
					
Initial State: N/A 
					
Input: Set of input values as V-Zero-Input-$b$ test case in Table \ref{tbl_inputValidation}.
					
Output: $x$, the position of the cart and $\theta$, the angle of the pendulum

Test Case Derivation: Derived from IP Simulator, as a normal and valid set of inputs.
How the test will be performed: Automatic test using Pytest.
\item{TC-V-Zero-Input-$f$: Valid and Zero Inputs}

Control: Automatic
					
Initial State: N/A 
					
Input: Set of input values as V-Zero-Input-$f$ test case in Table \ref{tbl_inputValidation}.
					
Output: $x$, the position of the cart and $\theta$, the angle of the pendulum

Test Case Derivation: Derived from IP Simulator, as a normal and valid set of inputs.			
How the test will be performed: Automatic test using Pytest.

\item{TC-Neg-Input: Negative Inputs}

Control: Automatic
					
Initial State: N/A 
					
Input: Set of input values as Neg-Input-$m_p$, Neg-Input-$m_c$, Neg-Input-$l_p$, Neg-Input-$f$, and Neg-Input-$b$ test cases in Table \ref{tbl_inputValidation}.
					
Output: Error message of invalid negative input.

Test Case Derivation: Inputs can not be negative values.
					
How the test will be performed: Automatic test using Pytest.

\item{TC-OutofBound: Out of Bound Inputs}

Control: Automatic
					
Initial State: N/A 
					
Input: Set of input values as OutofBound-$m_p$, OutofBound-$m_c$, and OutofBound-$l_p$ test cases in Table.\ref{tbl_inputValidation}.
					
Output: Error message of inputs because of constraints.

Test Case Derivation: The user should prepare values for inputs in the valid range for each.

How the test will be performed: Automatic test using Pytest.

\item{TC-Null-Input: Null Inputs}

Control: Automatic
					
Initial State: N/A 
					
Input: Set of input values as Null-Input-$m_c$, Null-Input-$m_p$, Null-Input-$l_p$, Null-Input-$f$, and Null-Input-$b$ test cases, in Table.\ref{tbl_inputValidation}.
					
Output: Error message of invalid missing input.

Test Case Derivation: The user should provide all the inputs without missing them.
					
					
How the test will be performed: Automatic test using Pytest.			
				
\end{enumerate}

\subsubsection{Output Validity \label{out_val}}

This section presents tests to make confidence in the out put and also error handling of the software, which means the system can generate the expected results from valid inputs and otherwise handle unexpected errors gracefully and provides clear error messages when errors occur.
The other goal of these tests is to verify that the system simulates the cart moving back and forth and that the pendulum rotates according to input values, as the R3 and R4 of requirements in section 5 of the  \href{https://github.com/MinMah23/CAS741-Project/tree/main/docs/SRS/SRS.pdf}{SRS} has defined.
All the outputs are measured at time equals to 3.000 seconds.

\begin{table}[ht]
\centering
\caption{Output Validation Test Cases} \label{tbl_output}
\vspace*{2mm}
 \begin{tabular}{|c|c c c c c| c c|} 
 \hline
   & &&\textbf{Input} &&  &\textbf{Expected Output}& \\ \hline
\textbf{TC} &   $m_p$ & $m_c$ & $l$ & $F$ & $b$  &    $Output$ &\\ \hline
TC-IP-Out-1 &0.2 & 1.5& 1.3& 50 & 0.1 &  $x$ = 109.000 & $\theta$ =  0.6984\\
 \hline
TC-IP-Out-2&2.5& 0.1&0.3&20&0& $x$ = 21.6257 & $\theta$ = 0.4034\\
 \hline
TC-IP-Out-3  &0.5& 4&1.3&2&0.2& $x$ =  11.0721 & $\theta$ = 0.9152\\
 \hline
TC-IP-Out-4 &0 & 51& -0.3& 50& -0.2 & Value Error &\\ \hline

\end{tabular}

\end{table}		


\begin{enumerate}

\item{TC-IP-Out-1}

Control: Automatic
					
Initial State: the software has run by the input values of TC-IP-Out-1 in Table \ref{tbl_output} and the initial conditions are $x_i$ = 0.2, $\dot{x_i} = 0$, $\theta$ = 0.3, $\dot{\theta{i}}=0$.
					
Input: input variables like TC-IP-Out-1 in Table \ref{tbl_output}.
					
Output: The $x$ will be 109.000  \si{\metre} and $\theta$ equals to 0.6984 \si{\radian}.\\
Test Case Derivation: Derived form IP Simulator, as a normal and valid set of inputs have been provided.\\					
How test will be performed: It will be performed by IP Simulator and Pytest.\\

\item{TC-IP-Out-2}

Control: Automatic
					
Initial State: the software has run by the input values of TC-IP-Out-2 in Table \ref{tbl_output} and the initial conditions are $x_i$ = 0, $\dot{x_i} = 0$, $\theta$ = 0.1, $\dot{\theta{i}}=0$.\\
Input: valid inputs like TC-IP-Out-2 in Table \ref{tbl_output}.\\	
Output: The $x$ will be 21.6257  \si{\metre} and $\theta$ equals to 0.4034 \si{\radian}.\\
How test will be performed:  It will be performed by IP Simulator, Pytest and Flake8.\\


\item{TC-IP-Out-3}

Control: Automatic
					
Initial State: the software has run by the input values of TC-IP-Out-3 in Table \ref{tbl_output} and the initial conditions are $x_i$ = 0, $\dot{x_i} = 0.2$, $\theta$ = 0.1, $\dot{\theta{i}}=0$.\\				
Input: valid inputs like TC-IP-Out-3 in Table \ref{tbl_output}.\\	
Output: The $x$ will be 11.0721 \si{\metre} and $\theta$ equals to 0.9152 \si{\radian}.\\
How test will be performed: It will be performed by IP Simulator, Pytest and Flake8.\\


\item{TC-IP-Out-4}

Control: Automatic
					
Initial State: the software has run by the input values of TC-IP-Out-4 in Table \ref{tbl_output} and the initial conditions are $x_i$ = 0, $\dot{x_i} = 0.2$, $\theta$ = 0.1, $\dot{\theta{i}}=0$.\\				
Input: invalid inputs like TC-IP-Out-4 in Table \ref{tbl_output}.\\	
Output: The Value Error will be raised.\\
How test will be performed: It will be performed by IP Simulator, Pytest and Flake8.

\end{enumerate}

\subsection{Tests for Nonfunctional Requirements}

\subsubsection{Portability\label{nfr1}}

\paragraph{Testing the portability of IP Simulator}

\begin{enumerate}

\item{T1: Portability}
\vspace{3mm}
\newline
Type: Manual
					
Initial State: None. 

Requirement ID(s): NFR-1					

Input/Condition: Executes all tests in \ref{inp_val} and \ref{out_val} in the IP Simulator in different operating systems, including Linux, Windows, and MacOS.
					
Output/Result: Successful test implies portability of the software.

How the test will be performed: The test will be performed manually by executing the software and tests in different OS, including Linux, Windows, and MacOS. A successful test is specified by the verification of
each test for functional requirements. 
				
\end{enumerate}
\subsubsection{Usability\label{nfr2}}

\paragraph{Testing the Usability of IP Simulator}

\begin{enumerate}

\item{T1: Usability}
\vspace{3mm}
\newline
Testing the usability will determine if the users have efficient interaction with the software and also the software is easy to use or not. The system will be tested against a usability test and survey which the details are in the Appendix \ref{appendix}.

Type: Manual
					
Initial State: The IP Simulator runs. 

Requirement ID(s): NFR-2			

Input/Condition: Any tests considering inputs criteria can be run by a group of individuals who have varying levels of experience with inverted pendulum systems and the software, including at least an undergraduate Physics student, a high school student, and the domain expert.
We will administer the questionnaire to the participants while they use the software and observe their behavior and note any issues they encounter.

Output/Result: According to the feedback from the survey, the usability of the software will be tested.

How the test will be performed: Manually, as it is a test of usability by the user, and thus requires the test to be run by the users.

\end{enumerate}				

\subsubsection{Accuracy\label{nfr3}}

\paragraph{Testing the Accuracy of IP Simulator}

\begin{enumerate}

\item{T1: Accuracy}
\vspace{3mm}
\newline
The accuracy test should be done by comparing the results with a similar system that is available on the Internet and has been developed using MATLAB. \cite{al-khazraji_2022} 

Type: Manual
					
Initial State: The IP Simulator and a similar project in MATLAB with the same inputs run.

Requirement ID(s): NFR-3			

Input/Condition: All of the tests in \ref{out_val} will be done.

Output/Result: Comparing the outputs of IP Simulator to the result of \cite{al-khazraji_2022} using the relative error. If the results from running the IP Simulator considered as IP Measured Value and the results of running \cite{al-khazraji_2022} as Actual Value, the relative error can be calculated as below:\\
$Absolute Error = |Actual Value - IP Measured Value|$\\
and then:\\
$Relative Error$ = $(\frac{Absolute Error}{Actual Value})$\\
\newline
The acceptable value for $Relative Error$ is  less than $0.01$.

How the test will be performed: Manually, as both projects should be run manually.

\end{enumerate}

\subsection{Traceability Between Test Cases and Requirements}
A traceability between test cases and requirements is shown in table \ref{tbl_rel}.


\begin{table} [ht]
\caption{Traceability Matrix between tests and requirements}\label{tbl_rel}
\vspace*{3mm}
\centering
\begin{tabular}{|c|c| c| c| c | c|c | c|}
\hline
Test Cases & R1 & R2 & R3& R4& NFR1& NFR2&NFR3  \\
 \hline   
 \ref{inp_val} &X &X & & &&& \\
 \hline
  \ref{out_val}   & & & X&X & && \\
 \hline
  \ref{nfr1}   && & & &  X&&\\
 \hline
  \ref{nfr2}  &&& & &  &X&\\
 \hline
  \ref{nfr3}  &&& & &  &&X\\
 \hline
\end{tabular}

\end{table}


\section{Unit Test Description\label{unittest}}


We will use \href{https://docs.pytest.org/en/7.2.x/}{Pytest} framework for unit, functional, and integration automated testing.
\subsection{Constant Parameter Module}
This module is suppose to store the constants for the software, including constraints on input values, simulation variables and gravity. The $constantM.py$ implements this requirement and using Pytest this module is tested and verified by running $test\_constantM.py$.\\
\subsection{Input Parameters Module}
This module is implemented in $inputM.py$, has three functions, the read\_inputs reads all the inputs from the file that users inserted, as described in section 6 in \href{https://github.com/MinMah23/CAS741-Project/tree/main/docs/Design/SoftDetailedDes}{MIS}. The constrains on the parameters are checked in verify\_inputs using Constant Parameter Module, and then if all the constraints are met, convert\_inputs prepares the parameters for the calculation process. $Test\_inputM.py$ checks the function of this module.\\
\subsection{Motion ODE Module}
$OdeM.py$ implements Motion ODE Module, and defines the motion equation for the cart and pendulum with their specification.\\
 
\subsection{ODE Solver Module}
ODEs defined in the motion ODE module in addition to initial conditions that Input Parameters Module reads, are passed to ODE Solver Module that is implemented in $odeSolverM.py$. In this module by using $Scipy$ library, the ODEs will be solved.\\
\subsection{Output Module}
$OutputM.py$ is responsible to verify the outputs and writes them to the output file and to do that, it has two functions, verify\_output, which checks the constrains on the generated outputs and write\_output\_to\_file function that gives the output file and the output and then writes output to the specified file.
 The output includes an array of time and corresponding values of position of the cart, velocity of the cart, the angle position of the pendulum, and the velocity of the pendulum.\\
\subsection{Plotting Module}
This module gives the time and the output of ODE solver module and shows the results as a graph in $plotM.py$
\subsection{IP Control Module}
IP control module runs all the modules in a right way in $mainM.py$. $Test\_mainM.py$ verifies the function of output, plotting modules together with all modules.

				
\bibliographystyle{plainnat}

\bibliography{ref}

\newpage

\section{Appendix\label{appendix}}

\subsection{Usability Survey Questions?}
To test usability of the IP Simulator, the Hinderks et al. questionnaire \cite{hinderks_schrepp_thomaschewski} will be used.
\newpage{}
\end{document}