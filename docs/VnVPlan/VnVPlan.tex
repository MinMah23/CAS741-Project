\documentclass[12pt, titlepage]{article}

\usepackage{booktabs}
\usepackage{tabularx}
\usepackage{hyperref}
\hypersetup{
    colorlinks,
    citecolor=blue,
    filecolor=black,
    linkcolor=red,
    urlcolor=blue
}
\usepackage[round]{natbib}

\input{}
\newcommand{\colZwidth}{1.0\textwidth}
\newcommand{\colAwidth}{0.13\textwidth}
\newcommand{\colBwidth}{0.82\textwidth}
\newcommand{\colCwidth}{0.1\textwidth}
\newcommand{\colDwidth}{0.05\textwidth}
\newcommand{\colEwidth}{0.8\textwidth}
\newcommand{\colFwidth}{0.17\textwidth}
\newcommand{\colGwidth}{0.5\textwidth}
\newcommand{\colHwidth}{0.28\textwidth}


\begin{document}

\title{System Verification and Validation Plan for IP Simulator: A software for simulating inverted pendulum system}
\author{Mina Mahdipour}
\date{\today}
	
\maketitle

\pagenumbering{roman}

\section{Revision History}

\begin{tabularx}{\textwidth}{p{4cm}p{2cm}X}
\toprule {\bf Date} & {\bf Version} & {\bf Notes}\\
\midrule
February 13, 2023 & 1.0 & The first version of VnV\\
February 13, 2023 & 1.1 & Added test cases\\
February 17, 2023 & 1.2 & Added tests for NFRs\\
\bottomrule
\end{tabularx}

\newpage

\tableofcontents

\listoftables

\listoffigures

\newpage

\section{Symbols, Abbreviations and Acronyms}

\renewcommand{\arraystretch}{1.2}
\begin{tabular}{l l} 
  \toprule		
  \textbf{symbol} & \textbf{description}\\
  \midrule 
 
  FR & Functional Requirements\\
  MIS & Management Information Systems\\
  NFR & Nonfunctional Requirements\\
  N/A & Not Applicable\\
  VnV & Verification and Validation\\
SRS & Software Requirements Specification\\
 T & Test\\
\bottomrule
\end{tabular}\\


\newpage

\pagenumbering{arabic}

This document outlines road map of the Verification and Validation plan for Inverted Pendulum (IP) Simulator to help ensure (but not prove) correctness and completeness of the program.
It includes a plan for testing the functional and non-functional requirements, an overview of the system tests, and an outline of the unit tests (not yet complete as it is depended
on the MIS).

\section{General Information}

\subsection{Summary}


This document provides the validation and verification plans for the IP Simulator Software. This software aims to simulate and describe the behaviour of the system of a cart-pendulum in the presence of different values of external force. To do that, it will calculate the cart position and pendulum's angle after applying the force. 

\subsection{Objectives}

The objectives of this document are listed below:
\begin{itemize}
\item Establish confidence in software correctness.
\item Demonstrate adequate usability.
\item Verify and validate the final product.
\end{itemize}

\subsection{Relevant Documentation}

\href{https://github.com/MinMah23/CAS741-Project/tree/main/docs/ProblemStatementAndGoals/ProblemStatement.pdf}{Problem Statement} presents an overview of the problem.
The requirements and an outline of the solution for the IP Simulator are captured in the \href{https://github.com/MinMah23/CAS741-Project/tree/main/docs/SRS/SRS.pdf}{Software Requirements Specification}.
The final report of running the test which described in this document, will be shown in \href{https://github.com/Maryamvalian/project741/blob/main/docs/VnVReport/VnVReport.pdf}{VnV Report}.
The software design information is captured in the \href{https://github.com/MinMah23/CAS741-Project/tree/main/docs/Design}{Module Guide}.

\section{Plan}
This section provides a road map of plans for different steps of producing the final product.
\subsection{Verification and Validation Team}
Our team includes:\\
\\

\noindent
\begin{minipage}{\textwidth}
\renewcommand*{\arraystretch}{1.5}
\begin{tabular}{| p{\colAwidth} | p{\colBwidth}|}
\hline

Roles& Responsibility \label{R&R}\\
\hline
Supervisor &\bf \\
\hline
% \hline
Author&  \\
\hline

% \hline
Reviewer&  \\
\hline

\end{tabular}
\end{minipage}\\



\subsection{SRS Verification Plan}

The SRS will be independently peer-reviewed by members of the VnV team according to \href{https://github.com/smiths/capTemplate/blob/main/docs/Checklists/SRS-Checklist.pdf}{SRS-Checklist}, consisting of the domain expert Deesha Patel and the SRS reviewer, Maryam Valian, as well as Dr. Spencer Smith.
The SRS document has been published to GitHub. Any issues identified during the review are tracked and verified in Github platform.

\subsection{Design Verification Plan}

The design document MIS will be reviewed by the MIS review team, consisting of the domain expert Deesha Patel and the MIS reviewer, Joachim de Fourestier, as well as Dr. Spencer Smith.
The MIS document will be published to GitHub. Defects will be addressed with issues
on the GitHub platform.
There is also \href{https://github.com/smiths/capTemplate/blob/main/docs/Checklists/MIS-Checklist.pdf}{MIS-Checklist}, that can be used.

\subsection{Verification and Validation Plan Verification Plan}
The VnV plan document will be reviewed by the VnV plan review team, consisting of the domain expert Deesha Patel and the reviewer, Karen Wang as well as Dr. Spencer Smith.
This document will be published to GitHub. Defects will be addressed with issues
on the GitHub platform.
There is also \href{https://github.com/smiths/capTemplate/blob/main/docs/Checklists/VnV-Checklist.pdf}{VnV-Checklist}, that can be used.

\subsection{Implementation Verification Plan}

The implementation of the IP Simulator will be tested under the test cases listed in this document including ,  , , ,.
\wss{In this section you would also give any details of any plans for static verification of
  the implementation.  Potential techniques include code walkthroughs, code
  inspection, static analyzers, etc.}

\subsection{Automated Testing and Verification Tools}

\wss{What tools are you using for automated testing.  Likely a unit testing
  framework and maybe a profiling tool, like ValGrind.  Other possible tools
  include a static analyzer, make, continuous integration tools, test coverage
  tools, etc.  Explain your plans for summarizing code coverage metrics.
  Linters are another important class of tools.  For the programming language
  you select, you should look at the available linters.  There may also be tools
  that verify that coding standards have been respected, like flake9 for
  Python.}

\wss{If you have already done this in the development plan, you can point to
that document.}

\wss{The details of this section will likely evolve as you get closer to the
  implementation.}

\subsection{Software Validation Plan}

\wss{If there is any external data that can be used for validation, you should
  point to it here.  If there are no plans for validation, you should state that
  here.}

\wss{You might want to use review sessions with the stakeholder to check that
the requirements document captures the right requirements.  Maybe task based
inspection?}

\wss{This section might reference back to the SRS verification section.}

\section{System Test Description}
	
\subsection{Tests for Functional Requirements}

\wss{Subsets of the tests may be in related, so this section is divided into
  different areas.  If there are no identifiable subsets for the tests, this
  level of document structure can be removed.}

\wss{Include a blurb here to explain why the subsections below
  cover the requirements.  References to the SRS would be good here.}

\subsubsection{Area of Testing1}

\wss{It would be nice to have a blurb here to explain why the subsections below
  cover the requirements.  References to the SRS would be good here.  If a section
  covers tests for input constraints, you should reference the data constraints
  table in the SRS.}
		
\paragraph{Title for Test}

\begin{enumerate}

\item{test-id1\\}

Control: Manual versus Automatic
					
Initial State: 
					
Input: 
					
Output: \wss{The expected result for the given inputs}

Test Case Derivation: \wss{Justify the expected value given in the Output field}
					
How test will be performed: 
					
\item{test-id2\\}

Control: Manual versus Automatic
					
Initial State: 
					
Input: 
					
Output: \wss{The expected result for the given inputs}

Test Case Derivation: \wss{Justify the expected value given in the Output field}

How test will be performed: 

\end{enumerate}

\subsubsection{Area of Testing2}

...

\subsection{Tests for Nonfunctional Requirements}

\wss{The nonfunctional requirements for accuracy will likely just reference the
  appropriate functional tests from above.  The test cases should mention
  reporting the relative error for these tests.  Not all projects will
  necessarily have nonfunctional requirements related to accuracy}

\wss{Tests related to usability could include conducting a usability test and
  survey.  The survey will be in the Appendix.}

\wss{Static tests, review, inspections, and walkthroughs, will not follow the
format for the tests given below.}

\subsubsection{Area of Testing1}
		
\paragraph{Title for Test}

\begin{enumerate}

\item{test-id1\\}

Type: Functional, Dynamic, Manual, Static etc.
					
Initial State: 
					
Input/Condition: 
					
Output/Result: 
					
How test will be performed: 
					
\item{test-id2\\}

Type: Functional, Dynamic, Manual, Static etc.
					
Initial State: 
					
Input: 
					
Output: 
					
How test will be performed: 

\end{enumerate}

\subsubsection{Area of Testing2}

...

\subsection{Traceability Between Test Cases and Requirements}

\wss{Provide a table that shows which test cases are supporting which
  requirements.}

\section{Unit Test Description}

\wss{Reference your MIS (detailed design document) and explain your overall
  philosophy for test case selection.}  
\wss{This section should not be filled in until after the MIS (detailed design
  document) has been completed.}

\subsection{Unit Testing Scope}

\wss{What modules are outside of the scope.  If there are modules that are
  developed by someone else, then you would say here if you aren't planning on
  verifying them.  There may also be modules that are part of your software, but
  have a lower priority for verification than others.  If this is the case,
  explain your rationale for the ranking of module importance.}

\subsection{Tests for Functional Requirements}

\wss{Most of the verification will be through automated unit testing.  If
  appropriate specific modules can be verified by a non-testing based
  technique.  That can also be documented in this section.}

\subsubsection{Module 1}

\wss{Include a blurb here to explain why the subsections below cover the module.
  References to the MIS would be good.  You will want tests from a black box
  perspective and from a white box perspective.  Explain to the reader how the
  tests were selected.}

\begin{enumerate}

\item{test-id1\\}

Type: \wss{Functional, Dynamic, Manual, Automatic, Static etc. Most will
  be automatic}
					
Initial State: 
					
Input: 
					
Output: \wss{The expected result for the given inputs}

Test Case Derivation: \wss{Justify the expected value given in the Output field}

How test will be performed: 
					
\item{test-id2\\}

Type: \wss{Functional, Dynamic, Manual, Automatic, Static etc. Most will
  be automatic}
					
Initial State: 
					
Input: 
					
Output: \wss{The expected result for the given inputs}

Test Case Derivation: \wss{Justify the expected value given in the Output field}

How test will be performed: 

\item{...\\}
    
\end{enumerate}

\subsubsection{Module 2}

...

\subsection{Tests for Nonfunctional Requirements}

\wss{If there is a module that needs to be independently assessed for
  performance, those test cases can go here.  In some projects, planning for
  nonfunctional tests of units will not be that relevant.}

\wss{These tests may involve collecting performance data from previously
  mentioned functional tests.}

\subsubsection{Module ?}
		
\begin{enumerate}

\item{test-id1\\}

Type: \wss{Functional, Dynamic, Manual, Automatic, Static etc. Most will
  be automatic}
					
Initial State: 
					
Input/Condition: 
					
Output/Result: 
					
How test will be performed: 
					
\item{test-id2\\}

Type: Functional, Dynamic, Manual, Static etc.
					
Initial State: 
					
Input: 
					
Output: 
					
How test will be performed: 

\end{enumerate}

\subsubsection{Module ?}

...

\subsection{Traceability Between Test Cases and Modules}

\wss{Provide evidence that all of the modules have been considered.}
				
\bibliographystyle{plainnat}

\bibliography{../../refs/References}

\newpage

\section{Appendix}

This is where you can place additional information.

\subsection{Symbolic Parameters}

The definition of the test cases will call for SYMBOLIC\_CONSTANTS.
Their values are defined in this section for easy maintenance.

\subsection{Usability Survey Questions?}

\wss{This is a section that would be appropriate for some projects.}

\newpage{}
\section*{Appendix --- Reflection}

The information in this section will be used to evaluate the team members on the
graduate attribute of Lifelong Learning.  Please answer the following questions:

\newpage{}
\section*{Appendix --- Reflection}

The information in this section will be used to evaluate the team members on the
graduate attribute of Lifelong Learning.  Please answer the following questions:

\begin{enumerate}
  \item What knowledge and skills will the team collectively need to acquire to
  successfully complete the verification and validation of your project?
  Examples of possible knowledge and skills include dynamic testing knowledge,
  static testing knowledge, specific tool usage etc.  You should look to
  identify at least one item for each team member.
  \item For each of the knowledge areas and skills identified in the previous
  question, what are at least two approaches to acquiring the knowledge or
  mastering the skill?  Of the identified approaches, which will each team
  member pursue, and why did they make this choice?
\end{enumerate}

\end{document}