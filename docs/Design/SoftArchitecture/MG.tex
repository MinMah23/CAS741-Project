\documentclass[12pt, titlepage]{article}

\usepackage{fullpage}
\usepackage[square,numbers]{natbib}
\usepackage[round]{natbib}
\usepackage{multirow}
\usepackage{booktabs}
\usepackage{tabularx}
\usepackage{graphicx}
\usepackage{float}
\usepackage{hyperref}
\hypersetup{
    colorlinks,
    citecolor=blue,
    filecolor=black,
    linkcolor=red,
    urlcolor=blue
}

\input{}

\newcounter{acnum}
\newcommand{\actheacnum}{AC\theacnum}
\newcommand{\acref}[1]{AC\ref{#1}}

\newcounter{ucnum}
\newcommand{\uctheucnum}{UC\theucnum}
\newcommand{\uref}[1]{UC\ref{#1}}

\newcounter{mnum}
\newcommand{\mthemnum}{M\themnum}
\newcommand{\mref}[1]{M\ref{#1}}

\begin{document}

\title{Module Guide for IP Simulator} 
\author{Mina Mahdipour}
\date{\today}

\maketitle

\pagenumbering{roman}

\section{Revision History}

\begin{tabularx}{\textwidth}{p{3cm}p{2cm}X}
\toprule {\bf Date} & {\bf Version} & {\bf Notes}\\
\midrule
March 1, 2023 & 1.0 & Created the document and fill initial data\\
March 9, 2023 & 1.1 & Updated modules and added module hierarchy diagram\\
March 17, 2023 & 1.2 & Merged two ODE modules and updated the document based on that.\\
March 21, 2023 & 1.3 & Updated according to feedback from reviewers.\\
April 15, 2023 & 1.4 & Updated and checked all the documents of the project\\
\bottomrule
\end{tabularx}

\newpage

\section{Reference Material}

This section records information for easy reference.

\subsection{Abbreviations and Acronyms}

\renewcommand{\arraystretch}{1.2}
\begin{tabular}{l l} 
  \toprule		
  \textbf{symbol} & \textbf{description}\\
  \midrule 
  AC & Anticipated Change\\
  IP & Inverted Pendulum\\
  M & Module \\
  MG & Module Guide \\
  ODE & Ordinary Differential Equation\\
  OS & Operating System \\
  R & Requirement\\
  SC & Scientific Computing \\
  SRS & Software Requirements Specification\\
  UC & Unlikely Change \\

  \bottomrule
\end{tabular}\\

\newpage

\tableofcontents

\listoftables

\listoffigures

\newpage

\pagenumbering{arabic}

\section{Introduction}

Decomposing a system into modules is a commonly accepted approach to developing
software.  A module is a work assignment for a programmer or programming
team \cite{ParnasEtAl1984}.  We advocate a decomposition based on the principle of information hiding \cite{Parnas1972a}.  This principle supports design for change, because the ``secrets'' that each module hides represent likely future changes.  Design for change is valuable in SC, where modifications are frequent, especially during initial development as the solution space is explored.  

Our design follows the rules layed out by \cite{ParnasEtAl1984}, as follows:
\begin{itemize}
\item System details that are likely to change independently should be the
  secrets of separate modules.
\item Each data structure is implemented in only one module.
\item Any other program that requires information stored in a module's data
  structures must obtain it by calling access programs belonging to that module.
\end{itemize}

After completing the first stage of the design, the Software Requirements
Specification (SRS), the Module Guide (MG) is developed \cite{ParnasEtAl1984}. The MG specifies the modular structure of the system and is intended to allow both designers and maintainers to easily identify the parts of the software.  The potential readers of this document are as follows:

\begin{itemize}
\item New project members: This document can be a guide for a new project member
  to easily understand the overall structure and quickly find the relevant modules they are searching for.
\item Maintainers: The hierarchical structure of the module guide improves the
  maintainers' understanding when they need to make changes to the system. It is
  important for a maintainer to update the relevant sections of the document
  after changes have been made.
\item Designers: Once the module guide has been written, it can be used to
  check for consistency, feasibility, and flexibility. Designers can verify the
  system in various ways, such as consistency among modules, feasibility of the
  decomposition, and flexibility of the design.
\end{itemize}

The author uses the \cite{Smith2016} as a reference. The rest of the document is organized as follows. Section
\ref{SecChange} lists the anticipated and unlikely changes of the software
requirements. Section \ref{SecMH} summarizes the module decomposition that
was constructed according to the likely changes. Section \ref{SecConnection}
specifies the connections between the software requirements and the
modules. Section \ref{SecMD} gives a detailed description of the
modules. Section \ref{SecTM} includes two traceability matrices. One checks
the completeness of the design against the requirements provided in the SRS. The other shows the relation between anticipated changes and the modules. Section \ref{SecUse} describes the use relation between modules.

\section{Anticipated and Unlikely Changes} \label{SecChange}

This section lists possible changes to the system. According to the likeliness
of the change, the possible changes are classified into two
categories. Anticipated changes are listed in Section \ref{SecAchange}, and
unlikely changes are listed in Section \ref{SecUchange}.

\subsection{Anticipated Changes} \label{SecAchange}

Anticipated changes are the source of the information that is to be hidden
inside the modules. Ideally, changing one of the anticipated changes will only
require changing the one module that hides the associated decision. The approach
adapted here is called design for change. Anticipated changes are numbered by \textbf{AC} followed by a number.

\begin{description}
\item[\refstepcounter{acnum} \actheacnum \label{acHardware}:] The specific
  hardware on which the software is running.
\item[\refstepcounter{acnum} \actheacnum \label{acInput}:] The format of the initial input data.
\item[\refstepcounter{acnum} \actheacnum \label{acParams}:] The format of the  input parameters.
\item[\refstepcounter{acnum} \actheacnum \label{acVerify}:] The constraints on
  the input parameters.
\item[\refstepcounter{acnum} \actheacnum \label{acOutput}:] The format of the  final output data.
\item[\refstepcounter{acnum} \actheacnum \label{acDS}:] The choice of array data structures used for storing and manipulating the data input and output.
\item[\refstepcounter{acnum} \actheacnum \label{acVerifyOut}:] The constraints on the output results.
\item[\refstepcounter{acnum} \actheacnum \label{acSolver}:] The algorithm used for solving the equations of motion.
\item[\refstepcounter{acnum} \actheacnum \label{acPlot}:] How the plotting of the output is implemented.
\item[\refstepcounter{acnum} \actheacnum \label{acControl}:] The overall control of the calculation.
\end{description}

\subsection{Unlikely Changes} \label{SecUchange}

The module design should be as general as possible. However, a general system is
more complex. Sometimes this complexity is not necessary. Fixing some design
decisions at the system architecture stage can simplify the software design. If
these decision should later need to be changed, then many parts of the design
will potentially need to be modified. Hence, it is not intended that these
decisions will be changed.

\begin{description}
\item[\refstepcounter{ucnum} \uctheucnum \label{ucIO}:] Input/Output devices (Input: File and/or Keyboard, Output: File, Memory, and/or Screen).
\item[\refstepcounter{ucnum} \uctheucnum \label{ucInput}:] There will always be a source of input data external to the software.
\item[\refstepcounter{ucnum} \uctheucnum \label{ucGoal}:] The goal of the system is to calculate the cart position and the pendulum angle.
\item[\refstepcounter{ucnum} \uctheucnum \label{ucODEstructureC}:] The equations of motion of the cart and the pendulum can be defined using parameters defined in the input parameters module.
\item[\refstepcounter{ucnum} \uctheucnum \label{ucEqM}:] How the equations of motion of the cart and the pendulum are defined using the specification of the system due to laws of Physics.
 
\end{description}

\section{Module Hierarchy} \label{SecMH}

This section provides an overview of the module design. Modules are summarized in a hierarchy decomposed by secrets in Table \ref{TblMH}. The modules listed below, which are leaves in the hierarchy tree, are the modules that will actually be implemented. Modules are numbered by \textbf{M} followed by a number. 

\begin{description}
\item [\refstepcounter{mnum} \mthemnum \label{mHH}:] Hardware-Hiding Module
\item [\refstepcounter{mnum} \mthemnum \label{mParams}:] Input Parameters Module
\item [\refstepcounter{mnum} \mthemnum \label{mCons}:] 
Constant Parameter Module
\item [\refstepcounter{mnum} \mthemnum \label{mOutput}:] Output Module
\item [\refstepcounter{mnum} \mthemnum \label{mODEC}:] Motion ODE  Module

\item [\refstepcounter{mnum} \mthemnum \label{mControl}:] IP Control Module
\item [\refstepcounter{mnum} \mthemnum \label{mSolver}:] ODE Solver Module
\item [\refstepcounter{mnum} \mthemnum \label{mPlot}:] Plotting Module
\item [\refstepcounter{mnum} \mthemnum \label{mDS}:]
Array Data Structure Module
\end{description}


Note that \mref{mHH} is a commonly used module and is already implemented by the operating system.  It will not be reimplemented.  Similarly, \mref{mSolver} , \mref{mPlot} and  \mref{mDS} are already available in Python and will not be reimplemented.



\begin{table}[h!]
\centering
\begin{tabular}{p{0.3\textwidth} p{0.6\textwidth}}
\toprule
\textbf{Level 1} & \textbf{Level 2}\\
\midrule

{Hardware-Hiding Module} & ~ \\
\midrule

\multirow{5}{0.3\textwidth}{Behaviour-Hiding Module} & Input Parameters Module\\
& Output Module\\
& Constant Parameter Module\\
&Motion ODE  Module\\
& IP Control Module\\
\midrule

\multirow{3}{0.3\textwidth}{Software Decision Module} & {ODE Solver Module}\\
& Array Data Structure Module\\
& Plotting Module\\
\bottomrule

\end{tabular}
\caption{Module Hierarchy}
\label{TblMH}
\end{table}

\section{Connection Between Requirements and Design} \label{SecConnection}

The design of the system is intended to satisfy the requirements described in the \href{https://github.com/MinMah23/CAS741-Project/tree/main/docs/SRS}{SRS}. In this stage, the system is decomposed into modules. The connection
between requirements and modules is listed in Table~\ref{TblRT}.


\section{Module Decomposition} \label{SecMD}

Modules are decomposed according to the principle of ``information hiding''
proposed by \citet{ParnasEtAl1984}. The \emph{Secrets} field in a module
decomposition is a brief statement of the design decision hidden by the
module. The \emph{Services} field specifies \emph{what} the module will do
without documenting \emph{how} to do it. For each module, a suggestion for the implementing software is given under the \emph{Implemented By} title. If the entry is \emph{OS}, this means that the module is provided by the operating system or by standard programming language libraries.  \emph{IP Simulator} means the module will be implemented by the IP Simulator software.

Only the leaf modules in the hierarchy have to be implemented. If a dash
(\emph{--}) is shown, this means that the module is not a leaf and will not have to be implemented.

\subsection{Hardware Hiding Modules (\mref{mHH})}

\begin{description}
\item[Secrets:]The data structure and algorithm used to implement the virtual hardware.
\item[Services:]Serves as a virtual hardware used by the rest of the
  system. This module provides the interface between the hardware and the
  software. So, the system can use it to display outputs or to accept inputs.
\item[Implemented By:] OS
\end{description}

\subsection{Behaviour-Hiding Module}

\begin{description}
\item[Secrets:] The contents of the required behaviours.
\item[Services:] Includes programs that provide externally visible behaviour of
  the system as specified in the \href{https://github.com/MinMah23/CAS741-Project/tree/main/docs/SRS}{SRS}
  documents. This module serves as a communication layer between the
  hardware-hiding module and the software decision module. The programs in this
  module will need to change if there are changes in the SRS.
\item[Implemented By:] --
\end{description}

\subsubsection{Input Parameters Module (\mref{mParams})}

\begin{description}
\item[Secrets:] The input data required for IP Simulator to run the simulation.

\item[Services:]This module reads input data from input file (including the mass of the pendulum, the mass of the cart, the length of the pendulum, the friction of the cart, the external force as a function of time, and the initial conditions) and stores them in the data structures. Then checks if the physical and software constraints are met. It throws a related error if inputs are not valid.
\item[Implemented By:] IP Simulator
\item[Type of Module:] Abstract Data Type
\end{description}


\subsubsection{Constant Parameter Module (\mref{mCons})}
\begin{description}
\item[Secrets:] The constant values used in the code.
\item[Services:] Stores all the constant values, including constraints on the input/output values that are mentioned in the table of specification parameters in the SRS document.

\item[Implemented By:] IP Simulator
\item[Type of Module:] Record
\end{description} 
\subsubsection{Output Module (\mref{mOutput})}

\begin{description}
\item[Secrets:] The format and structure of the output data.
\item[Services:] Outputs the results of the calculations, including the cart position and the pendulum angle over time and verifies that the output parameters comply with physical and software constraints.
% verify them and 
\item[Implemented By:] IP Simulator
\item[Type of Module:] Abstract Data Type
\end{description} 




\subsubsection{Motion ODE  Module (\mref{mODEC})}

\begin{description}
\item[Secrets:] The ODEs for finding the position of the cart and  the angle of the pendulum, using the input parameters. The equations are coupled to each other, so this module is responsible for defining both.
\item[Services:] Defines the ODEs using the parameters in the input parameters module.

\item[Implemented By:] IP Simulator
\item[Type of Module:] Abstract Object
\end{description} 


\subsubsection{IP Control Module(\mref{mControl})}

\begin{description}
\item[Secrets:] Execution flow of the IP Simulator. 
\item[Services:] Provides the main program and calls the different modules in the appropriate order.
\item[Implemented By:] IP Simulator
\item[Type of Module:] Abstract Object
\end{description} 


\subsection{Software Decision Module}

\begin{description}
\item[Secrets:] The design decision based on mathematical theorems, physical facts, or programming considerations. The secrets of this module are \emph{not} described in the SRS.
\item[Services:] Includes data structure and algorithms used in the system that do not provide direct interaction with the user. 
  % Changes in these modules are more likely to be motivated by a desire to
  % improve performance than by externally imposed changes.
\item[Implemented By:] --
\end{description}
\subsubsection{ODE Solver Module (\mref{mSolver})}

\begin{description}
\item[Secrets:] The algorithm to solve a system of first or higher order ODEs with initial values.
\item[Services:] Solves the ODEs using the governing equation, initial conditions, and numerical parameters.
\item[Implemented By:] Python
\item[Type of Module:] Library
\end{description}

\subsubsection{Plotting Module (\mref{mPlot})}

\begin{description}
\item[Secrets:] The data structures and algorithms for plotting the output data.
\item[Services:] Provides a plot function.
\item[Implemented By:] Python
\item[Type of Module:] Library
\end{description}

\subsubsection{ Array Data Structure Module (\mref{mDS})}

\begin{description}
\item[Secrets:] The data structure for manipulating and storing data types.
\item[Services:] Provides creating an array, reading a specific entry, manipulating, including building an array, and storing the data.
\item[Implemented By:] Python
\item[Type of Module:] Library
\end{description}

\section{Traceability Matrix} \label{SecTM}

Table \ref{TblRT} shows the traceability matrix between the modules and the
requirements. The traceability matrix between the modules and the anticipated changes has been shown in table \ref{TblACT}.

% the table should use mref, the requirements should be named, use something
% like fref
\begin{table}[H]
\centering
\begin{tabular}{p{0.2\textwidth} p{0.6\textwidth}}
\toprule
\textbf{Req.} & \textbf{Modules}\\
\midrule
R1 &  \mref{mHH}, \mref{mParams}, \mref{mControl}\\
R2 & \mref{mParams}, \mref{mCons}, \mref{mControl}\\
R3 & \mref{mCons}, \mref{mODEC}, \mref{mControl}, \mref{mSolver}, \mref{mDS}\\
R4 & \mref{mHH}, \mref{mOutput}, \mref{mControl}, \mref{mPlot}\\

\bottomrule
\end{tabular}
\caption{Trace Between Requirements and Modules}
\label{TblRT}
\end{table}


\begin{table}[H]
\centering
\begin{tabular}{p{0.2\textwidth} p{0.6\textwidth}}
\toprule
\textbf{AC} & \textbf{Modules}\\
\midrule
\acref{acHardware} & \mref{mHH}\\
\acref{acInput} & \mref{mParams}\\
\acref{acParams} &\mref{mParams}\\
\acref{acVerify} & \mref{mParams}, \mref{mCons}\\
\acref{acOutput} &  \mref{mOutput}\\
\acref{acDS} & \mref{mDS}\\
\acref{acVerifyOut} & \mref{mCons}, \mref{mOutput}\\
\acref{acSolver} & \mref{mSolver}\\
\acref{acPlot} & \mref{mPlot}\\
\acref{acControl} & \mref{mControl}\\
\bottomrule
\end{tabular}
\caption{Trace Between Anticipated Changes and Modules}
\label{TblACT}
\end{table}

\section{Use Hierarchy Between Modules} \label{SecUse}


In this section, the uses hierarchy between modules is provided. \cite{Parnas1978} said of two programs A and B that A {\em uses} B if correct execution of B may be necessary for A to complete the task described in its specification. That is, A {\em uses} B if there exist situations in which
the correct functioning of A depends upon the availability of a correct implementation of B.  Figure \ref{FigUH} illustrates the use relation between the modules. It can be seen that the graph is a tree. Each level of the hierarchy offers a testable and usable subset of the system, and modules in the higher level of the hierarchy are essentially simpler because they use modules from the lower levels.


%It can be seen that the graph is a directed acyclic graph (DAG). Each level of the hierarchy offers a testable and usable subset of the system, and modules in the higher level of the hierarchy are essentially simpler because they use modules from the lower levels.

\begin{figure}[H]
\centering
\includegraphics[width=0.7\textwidth]{hierarchy.jpg}
\caption{Use hierarchy among modules}
\label{FigUH}
\end{figure}

%\section*{References}

\bibliographystyle {plainnat}
\bibliography{reference.bib}

\newpage{}

\end{document}