\documentclass[12pt, titlepage]{article}

\usepackage{amsmath, mathtools}
\usepackage[square,numbers]{natbib}
\usepackage[round]{natbib}
\usepackage{amsfonts}
\usepackage{amssymb}
\usepackage{graphicx}
\usepackage{colortbl}
\usepackage{xr}
\usepackage{hyperref}
\usepackage{longtable}
\usepackage{xfrac}
\usepackage{tabularx}
\usepackage{float}
\usepackage{siunitx}
\usepackage{booktabs}
\usepackage{multirow}
\usepackage[section]{placeins}
\usepackage{caption}
\usepackage{fullpage}

\hypersetup{
bookmarks=true,     % show bookmarks bar?
colorlinks=true,       % false: boxed links; true: colored links
linkcolor=red,          % color of internal links (change box color with linkbordercolor)
citecolor=blue,      % color of links to bibliography
filecolor=magenta,  % color of file links
urlcolor=cyan          % color of external links
}

\usepackage{array}

\externaldocument{}

\input{}

\begin{document}

\title{Module Interface Specification for IP Simulator}

\author{Mina Mahdipour}

\date{\today}

\maketitle

\pagenumbering{roman}

\section{Revision History}

\begin{tabularx}{\textwidth}{p{3cm}p{2cm}X}
\toprule {\bf Date} & {\bf Version} & {\bf Notes}\\
\midrule
March 2, 2023 & 1.0 & First version created and added introduction parts\\
March 12, 2023 & 1.1 &  Added the modules\\
March 17, 2023 & 1.2 &  Updated ODE modules\\
March 21, 2023 & 1.3 &  Updated according to feedback of reviewers\\
April 16, 2023 & 1.4 &  Updated according to feedback of Prof.\ Smith\\
\bottomrule
\end{tabularx}

~\newpage

\section{Symbols, Abbreviations and Acronyms}

The reader can refer to \href{https://github.com/MinMah23/CAS741-Project/tree/main/docs/SRS}{SRS}. 

\newpage

\tableofcontents
\listoftables
%\listoffigures


\newpage

\pagenumbering{arabic}

\section{Introduction}

The following document details the Module Interface Specifications for IP Simulator software. This document specifies how every module is interfacing with every other part of the program based on "module state machine" approach.
\\Complementary documents include the \href{https://github.com/MinMah23/CAS741-Project/tree/main/docs/SRS}{System Requirement Specifications}
and \href{https://github.com/MinMah23/CAS741-Project/tree/main/docs/Design/SoftArchitecture}{Module Guide}.  The full documentation and implementation can be
found at the \href{https://github.com/MinMah23/CAS741-Project}{github repository for the IP Simulator}. The author uses \cite{Smith2016} as a reference to write this document.

\section{Notation}

The structure of the MIS for modules comes from \citet{HoffmanAndStrooper1995},
with the addition that template modules have been adapted from
\cite{GhezziEtAl2003}.  The mathematical notation comes from Chapter 3 of
\citet{HoffmanAndStrooper1995}.  For instance, the symbol := is used for a
multiple assignment statement and conditional rules follow the form $(c_1
\Rightarrow r_1 | c_2 \Rightarrow r_2 | ... | c_n \Rightarrow r_n )$.

The following table summarizes the primitive data types used by IP Simulator. 

\begin{center}
\renewcommand{\arraystretch}{1.2}
\noindent 
\begin{tabular}{l l p{7.5cm}} 
\toprule 
\textbf{Data Type} & \textbf{Notation} & \textbf{Description}\\ 
\midrule
character & char & a single symbol or digit\\
real & $\mathbb{R}$ & any number in (-$\infty$, $\infty$)\\
\bottomrule
\end{tabular} 
\end{center}

\noindent
The specification of IP Simulator uses some derived data types: sequences and strings. Sequences are lists filled with elements of the same data type and strings are sequences of characters. In addition, IP Simulator uses functions, which are defined by the data types of their inputs and outputs. Local functions are described by giving their type signature followed by their specification.

\section{Module Decomposition}

The following table is taken directly from the \href{https://github.com/MinMah23/CAS741-Project/blob/main/docs/Design/SoftArchitecture/MG.pdf}{Module Guide} document for this project.


\begin{table}[h!]
\centering
\begin{tabular}{p{0.3\textwidth} p{0.6\textwidth}}
\toprule
\textbf{Level 1} & \textbf{Level 2}\\
\midrule

{Hardware-Hiding Module} & ~ \\
\midrule

\multirow{5}{0.3\textwidth}{Behaviour-Hiding Module} & Input Parameters Module\\
& Output Module\\
& Constant Parameter Module\\
&Motion ODE  Module\\
& IP Control Module\\
\midrule

\multirow{3}{0.3\textwidth}{Software Decision Module} & {ODE Solver Module}\\
& Array Data Structure Module\\
& Plotting Module\\
\bottomrule

\end{tabular}
\caption{Module Hierarchy}
\label{TblMH}
\end{table}

\newpage


\section{ MIS of Input Parameters Module \label{MInput} }
This module is responsible for reading the input parameters from a file and storing them in the data structures. It also verifies them using Constant Parameter Module.

\subsection{Module}
InputM

\subsection{Uses}
ConstantM (Section \ref{MConstant})

\subsection{Syntax}



\subsubsection{Exported Constants}
None
\subsubsection{Exported Access Programs}

\begin{tabular}{p{3cm} p{2cm} p{2cm} >{\raggedright\arraybackslash}p{6cm}}
\toprule
\textbf{Name} & \textbf{In} & \textbf{Out} & \textbf{Exceptions} \\
\midrule
load\_inputs & string & - &  FileError \\
verify\_inputs & - & - & ValueError \\
$ m_p$ & - & $\mathbb{R}$\\
$l_p$ & - & $\mathbb{R}$\\
$m_c$ & - & $\mathbb{R}$\\
$friction$ & - & $\mathbb{R}$\\
$f\_external(t)$ & $\mathbb{R}$ & $\mathbb{R}$&\\
$i_p$& - & $\mathbb{R}$\\
$x_i$ & - & $\mathbb{R}$\\
$\dot{x_i}$ & - & $\mathbb{R}$\\
$\theta{i}$ & - & $\mathbb{R}$\\
$\dot{\theta{i}}$ & - & $\mathbb{R}$\\

\bottomrule
\end{tabular}
\subsection{Semantics}

\subsubsection{State Variables}
\#From R1:\\
$ m_p$: $\mathbb{R}$\\
$l_p$: $\mathbb{R}$\\
$m_c$: $\mathbb{R}$\\
$friction$: $\mathbb{R}$\\
$f\_external(t)$: t: $\mathbb{R} \rightarrow res: \mathbb{R}$\\
$i_p$: $\mathbb{R}$\\
$x_i$: $\mathbb{R}$\\
$\dot{x_i}$: $\mathbb{R}$\\
$\theta{i}$: $\mathbb{R}$\\
$\dot{\theta{i}}$: $\mathbb{R}$\\


\subsubsection{Environment Variables}

In the case of this module, the environment variable is the $fName$, the name of input file which is containing the input data.

\subsubsection{Assumptions}
\begin{itemize}

\item load\_inputs will be called before the values of any state variables will be accessed.

\item The file contains the expected inputs data in order, each on a new line. The order of the input data is as below:
\begin{itemize}
\item line 1: a string of "test case" to specify the start of test
\item line 2: the cart mass  
\item line 3: the pendulum mass   
\item line 4: the pendulum length
\item line 5: the friction of the cart
\item line 6: the external force as a function of time
\item line 7: the initial condition for the position of the cart
\item line 8: the initial condition for the velocity of the cart
\item line 9: the initial condition for the angle of the pendulum
\item line 10: the initial condition for the velocity of the pendulum
\item line 11: a string of "end" to specify the end of this test
\end{itemize}

\end{itemize}
\subsubsection{Access Routine Semantics}
The value of each state variable can be accessed through its name (getter). An access program is available for each state variable. There are no setters for the state variables, since the values will be set by load params and not changed for the life of the program.\\
\\
\noindent load\_inputs($fName$: string):
\begin{itemize}
\item transition: The filename $fName$ is associated with the input file. The state variables are modified with the following procedures:
\begin{enumerate}
\item Read data  from input file to populate the state variables from R1.
\item Store the data parameters.
\item Calculate the derived quantity as follows (from DD3 in SRS):\\
        $i_p$= $m_p(l_p^2)$ 
\item Verify the inputs through verify\_inputs().
\item Convert the format of inputs through convert\_inputs().
\end{enumerate}

\item exception: exc := a file name $fName$ cannot be found OR the format of input file is incorrect $\Rightarrow$  FileError
\end{itemize}

\noindent verify\_inputs():
\begin{itemize}
\item out: \textit{out} := none
\item exception: exc := 
\end{itemize}

\noindent \begin{longtable*}[l]{l l} 
$\neg (m_{\text{pmin}} \leq m_p \leq m_{\text{pmax}})$ & $\Rightarrow$ ValueError\\

$\neg (l_{\text{pmin}} \leq l_p \leq l_{\text{pmax}})$& $\Rightarrow$  ValueError\\
$\neg (m_{\text{cmin}} \leq m_c\leq m_{\text{cmax}})$ & $\Rightarrow$  ValueError \\
$\neg (0 \leq $friction$)$ & $\Rightarrow$ ValueError\\
$\neg  (0 < x_i)$ & $\Rightarrow$ ValueError\\
$\neg  (0 < \dot{x_i})$ & $\Rightarrow$ ValueError\\
$\neg  (0 < \theta{i}<2\pi)$ & $\Rightarrow$ ValueError\\
$\neg  (0 < \dot{\theta{i}})$ & $\Rightarrow$ ValueError\\
\end{longtable*}
etc.  See Appendix (Section~\ref{Appendix}) for the complete list of exceptions and associated error messages.\\
\newline
\noindent convert\_inputs():
\begin{itemize}
\item out: \textit{out} := none
\item exception: none 
\end{itemize}

\noindent InputM.$m_p$:
\begin{itemize}
\item output: \textit{out} := $m_p$
\item exception: none
\end{itemize}

\noindent InputM.$l_p$:
\begin{itemize}
\item output: \textit{out} := $l_p$
\item exception: none
\end{itemize}
\noindent InputM.$i_p$:
\begin{itemize}
\item output: \textit{out} := $i_p$
\item exception: none
\end{itemize}
\noindent InputM.$m_c$:
\begin{itemize}
\item output: \textit{out} := $m_c$
\item exception: none
\end{itemize}

\noindent InputM.$friction$:
\begin{itemize}
\item output: \textit{out} := $friction$
\item exception: none
\end{itemize}\noindent 

\noindent InputM.$f\_external(t)$:
\begin{itemize}
\item output: \textit{out} := $f\_external(t)$
\item exception: none
\end{itemize}\noindent

\noindent InputM.$x_i$:
\begin{itemize}
\item output: \textit{out} := $x_i$
\item exception: none
\end{itemize}

\noindent InputM.$\dot{x_i}$:
\begin{itemize}
\item output: \textit{out} := $\dot{x_i}$
\item exception: none
\end{itemize}

\noindent InputM.$\theta{i}$:
\begin{itemize}
\item output: \textit{out} := $\theta{i}$
\item exception: none
\end{itemize}

\noindent InputM.$\dot{\theta{i}}$:
\begin{itemize}
\item output: \textit{out} := $\dot{\theta{i}}$
\item exception: none
\end{itemize}



\newpage
\section{MIS of Output Module\label{MOutput} }


\subsection{Module}
OutputM

\subsection{Uses}
ConstantM (Section \ref{MConstant})

\subsection{Syntax}

\subsubsection{Exported Constants}
None
\subsubsection{Exported Access Programs}

\begin{center}
\begin{tabular}{p{2cm} p{8cm} p{2cm} p{2cm}}
\hline
\textbf{Name} & \textbf{In} & \textbf{Out} & \textbf{Exceptions} \\
\hline
 output& $fOName$: string, $t$$: \mathbb{R}^{n}$, $x$$: \mathbb{R}^{n}$, $\dot{x}$: $\mathbb{R}^{n}$, $\theta$: $\mathbb{R}^{n}$, $\dot{\theta}$$: 
 \mathbb{R}^{n}$ & - & FileError\\
verify\_out&  [$x$$: \mathbb{R}^{n}$, $\dot{x}$: $\mathbb{R}^{n}$, $\theta$: $\mathbb{R}^{n}$, $\dot{\theta}$$: 
 \mathbb{R}^{n}$] & - & InvalidOutput \\
\end{tabular}
\end{center}

\subsection{Semantics}

\subsubsection{State Variables}
None
\subsubsection{Environment Variables}
The environment variable is the $fOName$, the name of the output file which includes a sequence of angle of the pendulum and a sequence of position of the cart with regards to the data inputs during the time of simulation.

\subsubsection{Assumptions}
None
\subsubsection{Access Routine Semantics}


\noindent output ($fOName$: string, $t$$: \mathbb{R}^{n}$, $x$$: \mathbb{R}^{n}$, $\dot{x}$: $\mathbb{R}^{n}$, $\theta$: $\mathbb{R}^{n}$, $\dot{\theta}$$: 
 \mathbb{R}^{n}$):
\begin{itemize}

\item transition: The filename $fOName$ is associated with the outputFile. The inputs of this function are the sequence of $x$ and $\dot{x}$, the array of position and velocity of the cart while the sequence of  $\theta$ and $\dot{\theta}$ are the array of position and velocity of the pendulum at time $t$.
The below procedure is followed:
\begin{enumerate}
\item verify\_out ($x$$: \mathbb{R}^{n}$, $\dot{x}$: $\mathbb{R}^{n}$, $\theta$: $\mathbb{R}^{n}$, $\dot{\theta}$$: \mathbb{R}^{n}$)
\item write the calculated cart position, cart velocity, pendulum angle, and pendulum velocity for a specified time in the output file.
\end{enumerate}

\item exception: exc := a file name $fOName$ cannot be found $\Rightarrow$  FileError

\end{itemize}

\noindent verify\_out ($x$$: \mathbb{R}^{n}$, $\dot{x}$: $\mathbb{R}^{n}$, $\theta$: $\mathbb{R}^{n}$, $\dot{\theta}$$: \mathbb{R}^{n}$):
\begin{itemize}
\item output: none
\item exception: exec:= 
\end{itemize}

\noindent
\begin{longtable*}[l]{l l} 
\# for $ (0 \leq i \leq n)$, all items in the sequence of outputs: \\ 
$\neg  (0 < \theta < 2\cdot\pi)$& $\Rightarrow$ Invalid Angle the Pendulum\\
$\neg  (0 < x)$ & $\Rightarrow$ Invalid Velocity of the Pendulum\\
\end{longtable*}


\subsubsection{Local Functions}
None



\newpage
\section{MIS of Constant Parameter Module \label{MConstant}} 


\subsection{Module}

ConstantM

\subsection{Uses}

Not Applicable
\subsection{Syntax}


\subsubsection{Exported Constants}
\# From Table 2 in \href{https://github.com/MinMah23/CAS741-Project/tree/main/docs/SRS}{SRS}
\begin{center}
\begin{tabular}{p{2cm} p{2cm} p{2 cm} p{8cm}}
\hline

\textbf{Name} & \textbf{Type} & \textbf{Value} & \textbf{Description} \\
\hline
 $g$ & $\mathbb{R}$ & 9.81 \si {\metre\per\square\second}&the gravity of the earth\\
 $m_{\text{pmin}}$&$\mathbb{R}$&0.01 \si{\kg} &minimum value of the pendulum mass\\
$ m_{\text{pmax}}$&$\mathbb{R}$& 50 \si{\kg} &maximum value of the pendulum mass\\
$ l_{\text{pmin}}$&$\mathbb{R}$&0.01 \si{\m}&minimum value of the length of the pendulum\\
$ l_{\text{pmax}}$&$\mathbb{R}$&10 \si{\m}&maximum value of the length of the pendulum\\
$ m_{\text{cmin}}$&$\mathbb{R}$& 0.01 \si{\kg} &minimum value of the mass of the cart\\
$ m_{\text{cmax}}$&$\mathbb{R}$&50 \si{\kg} &maximum value of the mass of the cart\\

$t_\text{start}$&$\mathbb{R}$&0 \si {\second}&the start time of simulation\\
$t_\text{span}$&$\mathbb{R}$&20 \si {\second}&the duration time of simulation\\
 \hline

\end{tabular}
\end{center}
\subsubsection{Exported Access Programs}
None
\subsection{Semantics}

Not Applicable



\newpage
\section{MIS of Motion ODE  Module  \label{MC} }


\subsection{Module}

EqMo

\subsection{Uses}

ConstantM (Section \ref{MConstant})


\subsection{Syntax}

\subsubsection{Exported Constants}
None
\subsubsection{Exported Access Programs}

\begin{center}
\begin{tabular}{p{3cm} p{6cm} p{4cm} p{3cm}}
\hline
\textbf{Name} & \textbf{In} & \textbf{Out} & \textbf{Exceptions} \\
\hline
ODE\_Motion & $ m_p$: $\mathbb{R}$, $l_p$: $\mathbb{R}$, $m_c$: $\mathbb{R}$, $friction$: $\mathbb{R}$, $f\_external(t)$: string, $i_p$: $\mathbb{R}$, $x_i$$: \mathbb{R}$, $\dot{x_i}$: $\mathbb{R}$, $\theta{i}$: $\mathbb{R}$, $\dot{\theta{i}}$$: \mathbb{R}$ & $\dot{x}$: $\mathbb{R}$, $\ddot{x}$: $\mathbb{R}$, $\dot{\theta}$: $\mathbb{R}$, $\ddot{\theta}$: $\mathbb{R}$  & - \\
\end{tabular}
\end{center}

\subsection{Semantics}

\subsubsection{State Variables}
None
\subsubsection{Environment Variables}
None
\subsubsection{Assumptions}
None
\subsubsection{Access Routine Semantics}




\noindent ODE\_Motion($ m_p$: $\mathbb{R}$, $l_p$: $\mathbb{R}$, $m_c$: $\mathbb{R}$, $friction$: $\mathbb{R}$, $f\_external(t)$, $i_p$: $\mathbb{R}$, $x_i$$: \mathbb{R}$, $\dot{x_i}$: $\mathbb{R}$, $\theta{i}$: $\mathbb{R}$, $\dot{\theta{i}}$$: \mathbb{R}$):
\\

\noindent$\dot{x} = \dot{x} $\\
$\ddot{x} =\dfrac{- friction } {(m_c+m_p)}\dot{x}+ \dfrac{ f\_external(t)- m_pl_p\ddot{\theta}\cos{\theta}+ m_pl_p(\dot{\theta}) ^ 2 \sin{\theta}} {(m_c+m_p)}$\\

\noindent$\dot{\theta} = \dot{\theta}$\\
$\ddot{\theta}$= $\dfrac{m_pgl_p}{(i_p+m_pl_p^2)}\sin{\theta}-\dfrac{m_pl_p\ddot{x}}{(i_p+m_pl_p^2)}\cos{\theta}$\
\begin{itemize}
\item output:
[$\dot{x}$, $\ddot{x}$, $\dot{\theta}$, $\ddot{\theta}$]

\item exception: none
\end{itemize}

\subsubsection{Local Functions}
None




\newpage
\section{MIS of IP Control Module \label{MControl} }

\subsection{Module}
main


\subsection{Uses}
InputM (Section \ref{MInput}),
OutputM (Section \ref{MOutput}),
EqMo (Section \ref{MC}),
ODE Solver (Section \ref{MODE}),
Plot (Section \ref{MPlot}),

\subsection{Syntax}

\subsubsection{Exported Constants}

\subsubsection{Exported Access Programs}

\begin{center}
\begin{tabular}{p{2cm} p{4cm} p{4cm} p{2cm}}
\hline
\textbf{Name} & \textbf{In} & \textbf{Out} & \textbf{Exceptions} \\
\hline
main & - & - & - \\

\end{tabular}
\end{center}

\subsection{Semantics}
The Control Module is designed to control the process flow in the software. It organizes all other modules to satisfy all the requirements and also helps maintainability and expandability of IP Simulator by classifying different parts of the code.
\subsubsection{State Variables}
None
\subsubsection{Environment Variables}
None
\subsubsection{Assumptions}
None

\subsubsection{Access Routine Semantics}

\noindent main():
\begin{itemize}
\item transition: Control the flow input data, calculation, and the output data by following below steps: \\
\noindent Get ($fName$: string, $fOName$: string) from user\\
\noindent load\_inputs($fName$)\\
\noindent verify\_inputs()\\
\noindent ODE\_Motion($ m_p$: $\mathbb{R}$, $l_p$: $\mathbb{R}$, $m_c$: $\mathbb{R}$, $friction$: $\mathbb{R}$, $f\_external(t)$, $i_p$: $\mathbb{R}$, $x_i: \mathbb{R}$, $\dot{x_i}$: $\mathbb{R}$, $\theta{i}$: $\mathbb{R}$, $\dot{\theta{i}}$$: \mathbb{R}$)\\
\noindent solveODE($ODE\_Motion$, $ m_p: \mathbb{R}$, $l_p: \mathbb{R}$, $m_c: \mathbb{R}$, $friction: \mathbb{R}$, $f\_external(t)$, $i_p: \mathbb{R}$, $x_i: \mathbb{R}$, $\dot{x_i}: \mathbb{R}$, $\theta{i}: \mathbb{R}$, $\dot{\theta{i}}: \mathbb{R}$,  $t_\text{start}: \mathbb{R}$$, $[$x_i$, $\dot{x_i}$, $\theta{i}$, $\dot{\theta{i}}$]$: \mathbb{R}^4, t_\text{span}: \mathbb{R}$)\\
\noindent verify\_out($x$$: \mathbb{R}^{n}$, $\dot{x}$: $\mathbb{R}^{n}$, $\theta$: $\mathbb{R}^{n}$, $\dot{\theta}$$: \mathbb{R}^{n}$)\\
\noindent plot($t$: $\mathbb{R}^{n}$, $x$$: \mathbb{R}^{n}$, $\theta$: $\mathbb{R}^{n}$)\\
\noindent output($fOName$, $x$$: \mathbb{R}^{n}$, $\dot{x}$: $\mathbb{R}^{n}$, $\theta$: $\mathbb{R}^{n}$, $\dot{\theta}^{n}$$: \mathbb{R}^{n}$)\\
\item exception: none
\end{itemize}

\subsubsection{Local Functions}
None


\newpage

\section{MIS of ODE Solver Module \label{MODE}}

\subsection{Module}
ODE Solver
\subsection{Uses}

None
\subsection{Syntax}

\subsubsection{Exported Constants}
None
\subsubsection{Exported Access Programs}

\begin{center}
\begin{tabular}{p{2.5cm} >{\raggedright\arraybackslash}p{8cm} >{\raggedright\arraybackslash}p{2.43cm} p{2cm}}
  \hline
  \textbf{Name} & \textbf{In} & \textbf{Out} & \textbf{Except.} \\
  \hline
  solveODE & $ODE\_Motion: (\mathbb{R}^{n} \rightarrow \mathbb{R}^n)$, [$x_i$, $\dot{x_i}$, $\theta{i}$, $\dot{\theta{i}}$]: $\mathbb{R}^4$, $t_\text{start}: \mathbb{R}$, $t_\text{span}: \mathbb{R}$ & $\textbf{y}$: [$x: \mathbb{R}$, $\dot{x}: \mathbb{R}$, $\theta: \mathbb{R}$, $\dot{\theta}: \mathbb{R}$] & ODE\_Error\\



 
\end{tabular}
\end{center}


\subsection{Semantics}

\subsubsection{State Variables}
None
\subsubsection{Environment Variables}
None
\subsubsection{Assumptions}
None

\subsubsection{Access Routine Semantics}

\noindent solveODE($ODE\_Motion$, $ m_p: \mathbb{R}$, $l_p: \mathbb{R}$, $m_c: \mathbb{R}$, $friction: \mathbb{R}$, $f\_external(t)$, $i_p: \mathbb{R}$, $x_i: \mathbb{R}$, $\dot{x_i}: \mathbb{R}$, $\theta{i}: \mathbb{R}$, $\dot{\theta{i}}: \mathbb{R}$,  $t_\text{start}: \mathbb{R}$$, $[$x_i$, $\dot{x_i}$, $\theta{i}$, $\dot{\theta{i}}$]$: \mathbb{R}^4, t_\text{span}: \mathbb{R}$):\\ 
\begin{itemize}
\item output: $out := \textbf{y}(t)$ where 

$$\textbf{y}(t) = \textbf{y}_\text{iniCond} + \int_{t_\text{start}}^{t_\text{span}} \textbf{f}(s, \textbf{y}(s)) ds$$ and
$\textbf{y}_\text{iniCond}$ = [$x_i$, $\dot{x_i}$, $\theta{i}$, $\dot{\theta{i}}$]\\

 $\textbf{f}$ is the $ODE\_Motion$, the function in module EqMo \ref{MC}. $y(t)$ is calculated from $t = t_\text{start}$ to $t = t_\text{span}$, and the $\textbf{y}_\text{iniCond}$ is the initial conditions for solving the ODE.\\
We have two coupled ODEs in the IP software, which the first one describes the motion of the cart and the second one describes the motion of the pendulum.

\item exception: $exc :=$ ( ODE Solver Fails $\Rightarrow$ ODE\_ERR)
\end{itemize}

\subsubsection{Local Functions}
None
\newpage


\section{MIS of Plotting Module \label{MPlot}}


\subsection{Module}
Plot
\subsection{Uses}
Not Applicable

\subsection{Syntax}

\subsubsection{Exported Constants}
None
\subsubsection{Exported Access Programs}
\begin{center}
\begin{tabular}{p{2cm} p{4cm} p{2cm} p{2cm}}
\hline
\textbf{Name} & \textbf{In} & \textbf{Out} & \textbf{Exceptions} \\
\hline
plot & $t$: $\mathbb{R}^{n}$, $x$$: \mathbb{R}^{n}$, $\theta$: $\mathbb{R}^{n}$ & - & - \\

\end{tabular}
\end{center}
\subsection{Semantics}

\subsubsection{State Variables}
None
\subsubsection{Environment Variables}
win: 2D diagram displayed on the screen
\subsubsection{Assumptions}


\subsubsection{Access Routine Semantics}

\noindent  plot($t$: $\mathbb{R}^{n}$, $x$$: \mathbb{R}^{n}$, $\theta$: $\mathbb{R}^{n}$):
\begin{itemize}
\item transition: Modify win to display a plot where the vertical axis
  is the angle of the pendulum and the position of the cart.  The time should run from $t_\text{start}$ to $t_\text{span}$.
\item exception: none
\end{itemize}


\subsubsection{Local Functions}
None
\newpage



\bibliographystyle {plainnat}
\bibliography {reference.bib}

\newpage

\section{Appendix} \label{Appendix}
\begin{longtable}{l p{9cm}}
\caption{Possible Exceptions} \\
\toprule
\textbf{Message ID} & \textbf{Error Message} \\
\midrule
FileError & Error: The expected file does not exist. \\

ValueError & Error: The input is not valid. \\

InvalidOutput& Error: The output is not valid.\\
ODE\_Error&Error: When fails to solve the ODE.\\
\bottomrule
\end{longtable}
\end{document}